\documentclass[a4paper,12pt,titlepage]{article}

\usepackage[french]{babel}
\usepackage[latin1]{inputenc}

\title{BIBLIOTHEQUES  \bf librpsld\rm\  et \bf libfminigraf\rm}
\author{Laboratoire  National Saturne\\
        Groupe Informatique\\
        J.L.Hamel}
\date{\today}

\textwidth 16.8cm
\textheight 23.7cm
\oddsidemargin -.25in
\evensidemargin -.25in
\topmargin -1in
\topskip 1.5cm
\footskip 1.5cm
%\footheight 1.5cm
\headheight 1.5cm
\headsep 0cm
\sloppy

\newcommand{\darg}[1]{\item[\tt #1\rm]}
\newcommand{\fsub}[1]{\hbox {\tt SUBROUTINE #1  } \medskip }
\newcommand{\ffun}[1]{\hbox {\tt FUNCTION   #1  } \medskip }

\newenvironment{argdesc}{\begin{list}{-}{\leftmargin 8em \labelwidth 7em}}%
{\end{list}}


\begin{document}

\maketitle

\tableofcontents
\newpage

\section{G\'en\'eralit\'es}
%--------------------------

La pr\'esente notice d\'ecrit la version UNIX actuelle des biblioth\`eques
FORTRAN \tt lib\-fmini\-graf\rm\  (graphique sur consoles FALCO ou fen\^etres
<< xterm >>) et
\tt librpsld\rm\  (entr\'ee de donn\'ees au clavier) utilisables au L.N.S. sur
stations SUN et HP.

Ces biblioth\`eques assurent autant que possible la compatibilit\'e avec les
versions tournant sur MAC68K, ATARI ST, et PC.

En outre, elles comprennent d'importants perfectionnements par rapports aux
versions pr\'e\-c\'e\-dentes.

\section{Routines graphiques}
%----------------------------

    Les   modules  graphiques  pr\'esent\'es  ici  sont  con\c{c}us   pour 
faciliter  le  trac\'e de courbes dans des syst\`emes  de  coordonn\'ees 
vari\'es et permettre le transport de programmes graphiques sur
diff\'erentes machines. 

   La version pr\'esent\'ee ici fonctionne sous UNIX dans un environnement
graphique de type TEKTRONIX 4014 utilisable sur terminaux FALCO et sur
fen\^etres << xterm >> de terminaux X. Elle est impl\'ement\'ee sur SUN et
HP.

    On utilise un \'ecran graphique virtuel de $1024 \times 768$ points par
d\'efaut, compatible TEKTRONIX. \em Afin d'assurer la portabilit\'e de
programmes \'ecrits pour d'autres versions de ces biblioth\`eques (par
exemple les versions pour modules graphiques MITV et MIVG, ou pour PC), il
est possible de red\'efinir le nombre de pixels de l'\'ecran virtuel. \em
Lorsque la pr\'esente notice parle \em d'unit\'es << \'ecran >>\em, il s'agit des
unit\'es de \em l'\'ecran virtuel\em\ ; la biblioth\`eque assure la conversion en
points d'\'ecrans r\'eels de fa\c{c}on transparente pour l'utilisateur.

\begin{quote}
    {\bf Important ! } Tout programme utilisant les routines graphiques
doit commencer par un appel \`a \tt INISCA\rm\ (si n\'ecessaire) pour
d\'efinir l'\'ecran virtuel suivi d'un appel \`a \tt INIVCF\rm\ et se
terminer par un appel \`a \tt FINVCF\rm.
\end{quote}

\subsection{D\'efinition de l'\'ecran virtuel : \tt INISCA}
\fsub{INISCA(NPX,NPY)}

  \em  Red\'efinition des dimensions de l'\'ecran virtuel.\em
\begin{argdesc}
    \darg{NPX,NPY :} Entiers contenant les \em nombres de pixels moins
		      un\em\ horizontaux et verticaux de l'\'ecran virtuel. 

                 Valeurs par d\'efaut : \tt NPX\rm\ = 1023, \tt NPY\rm\  = 779
\end{argdesc}

\subsection{Init. du syst\`eme graphique : \tt INIVCF}
\fsub{INIVCF}

   \em Initialisation  du syst\`eme graphique (au  premier  appel),  et 
restitution des param\`etres par d\'e\-faut.\em
\begin{itemize}
              \item L'effet de limitation d'une fen\^etre est annul\'e, mais 
              non  la  correspondance  entre l'espace  physique et l'espace 
              \'ecran.

              \item Les  options  par  d\'efaut  pour  les   textes,   les 
              marqueurs et le type de ligne sont r\'etablies.

              \item Cette  routine doit \^etre appel\'ee au moins  une  fois 
              avant  tout appel de fonction graphique (sous  peine 
              de non fonctionnement ou de plantage).
\end{itemize}

\subsection{Arr\^et du syst\`eme graphique : \tt FINVCF}
\fsub{FINVCF}

    \em L'appel \`a cette routine met fin \`a l'utilisation des fonctions
    graphiques.\em

\subsection{Commutation alphanum\'erique vers graphique : \tt TXTFBG}
\fsub{TXTFBG}

   \em Commutation de l'\'ecran alphanum\'erique vers l'\'ecran graphique.\em

   Cette routine, invoqu\'ee automatiquement par les fonctions de trac\'e de
vecteur et de texte ne devrait pas normalement \^etre appel\'ee directement.

\subsection{Commutation graphique vers alphanum\'erique : \tt FBGTXT}
\fsub{FBGTXT}

   \em Commutation de l'\'ecran graphique vers l'\'ecran alphanum\'erique.\em

   Cette routine doit \^etre appel\'ee chaque fois que l'on d\'esire
\'ecrire sur l'\'ecran alphanum\'erique apr\`es une op\'eration graphique.
Elle est cependant invoqu\'ee automatiquement par les routines \tt RPSLDS\rm\
et \tt FINVCF\rm. 

\subsection{Trac\'e d'axes automatique : \tt TRAXE}
\fsub{TRAXE(XMIN,XMAX,YMIN,YMAX,MODE)}

    \em Routine de trac\'e d'axes automatique.\em

\begin{argdesc}
    \darg{XMIN,XMAX,YMIN,YMAX  :}
              R\'eels ;  bornes de la fen\^etre en unit\'es physiques.
    \darg{MODE  :}      Entier  de un ou deux chiffres d\'ecimaux de la  forme 
              $xy$ ;   d\'efinit  le type des  axes.  Signification  des 
              chiffres $x$ et $y$ ($x$ peut \^etre absent) : 

              \begin{center}\begin{tabular}{ll}
            $x$  &  0 ou absent  :  axes lin\'eaires.\\
                 &  1  :  axe X log.    , axe Y lin\'eaire.\\
                 &  2  :  axe X lin\'eaire, axe Y log.\\
                 &  3  :  axes logarithmiques.\\
            $y$  &  1  :  axes normaux sans grille.\\
                 &  2  :  axes avec grille.\\
                 &  3  :  axes non trac\'es.\\
              \end{tabular}\end{center}
\end{argdesc}

\subsection{Valeur des divisions sur un axe : \tt TRGETD}
\fsub{TRGETD(VMIN,VMAX,DIV,TIC)}

    \em Routine donnant, en unit\'es physiques, la valeur des grandes et
petites divisions sur un axe.\em

    Les arguments sont des r\'eels exprim\'es en unit\'es physiques.

\begin{argdesc}
    \darg{VMIN,VMAX :} Valeurs d'entr\'ee : minimum et maximum sur l'axe.
    \darg{DIV~,TIC~ :} Valeurs retourn\'ees : grande et petite division.
\end{argdesc}

\subsection{D\'efinition de fen\^etre graphique : \tt TRDEF}
\fsub{TRDEF(XMIN,XMAX,YMIN,YMAX,IMIN,IMAX,JMIN,JMAX)}

    \em Routine d\'efinissant la correspondance entre une fen\^etre physique
et une fen\^etre en unit\'es de l'\'ecran virtuel. Un clipping est
effectu\'e, limitant un dessin \`a la fen\^etre active.\em

   Cette routine est utilis\'ee en particulier par \tt TRAXE\rm.

\begin{argdesc}
    \darg{XMIN,XMAX,YMIN,YMAX :}
              R\'eels ;  bornes de la fen\^etre en unit\'es physiques.
    \darg{IMIN,IMAX,JMIN,JMAX :}
              Entiers ;   bornes  de  la fen\^etre  correspondante  en 
              unit\'es de l'\'ecran.
\end{argdesc}

\subsection{Taille des caract\`eres graphiques : \tt DEFCAR}
\fsub{DEFCAR(IHAUT,IFORM,IEFFA)}

    \em Routine d\'efinissant la taille des textes \'ecrits en caract\`eres
graphiques par la routine\em\ \tt TRTEXT\rm\em .\em

   Dans cette version, cette routine est limit\'ee aux possibilit\'es des
terminaux TEKTRONIX. Seul, le premier argument est utilis\'e, les autres ont
\'et\'e conserv\'es pour rester compatible avec les autres versions.

\begin{argdesc}
    \darg{IHAUT :}     Entier  de 1 \`a 4 d\'efinissant la  taille.  
              La  taille ordinaire par d\'efaut est 2.
        \begin{center}\begin{tabular}{|c|c|c|}
        \hline
     \tt IHAUT\rm     & Lignes & Colonnes\\
        \hline
           1 & 58     & 133 \\
           2 & 50     & 120 \\
           3 & 38     & ~81 \\
           4 & 35     & ~76 \\
     \hline
        \end{tabular}\end{center}
    \darg{IFORM :} Entier non utilis\'e dans cette version (mettre 0).
    \darg{IEFFA :} Entier non utilis\'e dans cette version (mettre 0).
\end{argdesc}

\subsection{D\'efinition des marqueurs : \tt DEFMKR}
\fsub{DEFMKR(LHAUT,ITYPE)}

    \em Routine  d\'efinissant  les marqueurs des  points  d'une  courbe 
trac\'ee en mode point (option 9 de la routine\em\ \tt  INIVPH\rm \em ).\em

\begin{argdesc}
    \darg{LHAUT :} Entier ;    hauteur   en  nombre  de  points   d'\'ecran 
              (ineffectif si \tt ITYPE\rm\ = 1 ).
    \darg{ITYPE :}  Entier d\'efinissant le type de marqueur. 

              \begin{center}\begin{tabular}{ll}
              1 &  point d'\'ecran \'el\'ementaire (option par d\'efaut).\\
              2 &   croix droite.\\
              3 &   ast\'erisque.\\
              4 &   carr\'e droit.\\
              5 &   croix diagonale.\\
              6 &   carr\'e diagonal.\\
              \end{tabular}\end{center}
\end{argdesc}

\subsection{Ecriture de texte graphique : \tt TRTEXT}
\fsub{TRTEXT(X,Y,TEXTE,IUNIT)}

    \em Routine d'\'ecriture de texte graphique.\em

\begin{argdesc}
    \darg{X,Y :} R\'eels ;   coordonn\'ees  du  coin  inf\'erieur  gauche  du 
              premier  caract\`ere du texte en unit\'es  physi\-ques  ou 
              \'ecran selon l'argument \tt IUNIT\rm.
    \darg{TEXTE :} Cha\^{\i}ne ou sous cha\^{\i}ne de caract\`eres 
                (type \tt CHARACTER\rm ) 
              contenant le texte, ou bien texte entre quotes.
    \darg{IUNIT :} Entier  d\'efinissant le type des unit\'es pour 
                           \tt X\rm\ et  \tt Y\rm. 

              \begin{center}\begin{tabular}{ll}
              0 &   unit\'es \'ecran.\\
              1 &   unit\'es physiques.\\
              \end{tabular}\end{center}
\end{argdesc}

\subsection{Type de ligne et mode point : \tt  INIVPH}
\fsub{INIVPH(NTYPE)}

    \em D\'efinition du type de ligne utilis\'e lors des trac\'es.\em

\begin{argdesc}
    \darg{NTYPE :} Entier ;  d\'efinit le type de ligne.

              \begin{itemize}
              \item[1]    trait continu (option par d\'efaut).
              \item[2]    pointill\'es
              \item[3]    pointill\'es et tirets
              \item[4]    tirets courts
              \item[5]    tirets longs
              \item[6]    inutilis\'e
              \item[7]    inutilis\'e
              \item[8]    inutilis\'e
              \item[9]    mode  point :   dans ce mode seul le point ou  le 
                   marqueur situ\'e \`a l'extr\'emit\'e du vecteur est
		   trac\'e (Cf. routine \tt DEFMKR\rm).
              \end{itemize}
\end{argdesc}

\subsection{Trac\'e  de  vecteur (espace physique) : \tt VECTPH}
\fsub{VECTPH(X,Y,MODE)}

    \em Trac\'e  de  vecteur en unit\'es physiques.\em

  Pour utiliser cette routine, la correspondance entre les unit\'es
\'ecran et les unit\'es physiques doit avoir \'et\'e d\'efinie par au
moins un appel \`a \tt TRAXE\rm\ ou \tt TRDEF\rm.

\begin{argdesc}
    \darg{X,Y :} R\'eels  repr\'esentant  les coordonn\'ees d'un  point  de 
              l'espace physique. A chaque appel, le sous programme 
              prend  pour origine du vecteur le point  d\'efini  par 
              l'appel pr\'ec\'edent, et pour extr\'emit\'e le point d\'efini 
              par l'appel courant.
    \darg{MODE :}      Entier d\'efinissant le mode de trac\'e.

              \begin{itemize}
              \item[0]    Effacement  de  l'\'ecran  et  positionnement  du 
                   curseur au point (\tt X,Y\rm ) sans trac\'e.
              \item[1]    Trac\'e en mode effacement (non impl\'ement\'e).
              \item[2]    Trac\'e en mode normal (opaque).
              \item[3]    Trac\'e  en  mode inversion (non impl\'ement\'e).
              \item[4]    Positionnement au point (\tt X,Y\rm ) sans trac\'e.
              \end{itemize}
\end{argdesc}

\subsection{Trac\'e de vecteur (espace \'ecran) : \tt VECTF}
\fsub{VECTF(IX,IY,MODE)}

    \em Trac\'e de vecteur en unit\'es \'ecran.\em

\begin{argdesc}
    \darg{IX,IY :} Entiers ;   coordonn\'ees du point courant,  comme  pour 
              \tt VECTPH\rm ,  mais en unit\'es \'ecran. 

    \darg{MODE :} Entier ; mode de trac\'e ; m\^emes sp\'ecifications que
              pour \tt VECTPH\rm , ci-dessus.
\end{argdesc}

\subsection{Trac\'e de symboles : \tt TRSYMB}
\fsub{TRSYMB(X,Y,MODE,ISYMB,ITYPX,ITYPY,COEFX,COEFY)}

    \em Trac\'e  de symboles (d\'efinis par l'utilisateur)  dans  l'espace 
physique.\em

\begin{argdesc}
    \darg{X,Y :} R\'eels  d\'efinissant  les coordonn\'ees  de  l'\'ecran  en 
              unit\'es physiques. 
    \darg{MODE :}  Entier ;  mode de trac\'e.

              \begin{center}\begin{tabular}{ll}
              1 &   mode effacement (non impl\'ement\'e).\\
              2 &   mode normal.\\
              \end{tabular}\end{center}
    \darg{ISYMB :} Tableau de type \tt INTEGER*2\rm\ d\'efinissant le symbole. Il 
              s'agit d'une suite de listes de la forme :

                   \centerline{$nbpoints,x_1,y_1,x_2,y_2,...\;x_n,y_n$}
	      \noindent suivie d'un z\'ero, d\'ecrivant les trac\'es
	      composant le symbole.  Les coordonn\'ees $x_i,y_i$ sont d\'efinies
	      par rapport \`a un syst\`eme d'axes li\'e au symbole, avec
	      des unit\'es arbitraires.  Les coefficients multiplicatifs
	      \tt COEFX, COEFY\rm\ permettent le passage de ces unit\'es vers les
	      unit\'es physiques ou \'ecran, suivant le cas. 
    \darg{ITYPX :} Entier ;  indique l'espace utilis\'e pour le coefficient 
              \tt COEFX\rm.

              \begin{center}\begin{tabular}{ll}
              0 &   espace \'ecran.\\
              1 &   espace physique.\\
              \end{tabular}\end{center}
    \darg{ITYPY :} Entier ;  indique l'espace utilis\'e pour le coefficient 
              \tt COEFY\rm.

              \begin{center}\begin{tabular}{ll}
              0 &   espace \'ecran.\\
              1 &   espace physique.\\
              \end{tabular}\end{center}
    \darg{COEFX :} R\'eel ;  coefficient de grandissement horizontal.
    \darg{COEFY :} R\'eel ;  coefficient de grandissement vertical.
\end{argdesc}

\subsection{Copie d'\'ecran sur fichier ou imprimante : \tt SAVECR}
\fsub{SAVECR(NOMFIC)}

    \em Sauvegarde de l'\'ecran sur un fichier au format PostScript en vue
d'une r\'eutilisation ult\'e\-rieure (pour l'incorporer dans un document par
exemple) ou copie d'\'ecran sur imprimante.\em
\begin{argdesc}
    \darg{NOMFIC :} Deux possibilit\'es :
       \begin{description}
          \item[Sortie sur fichier : ] Cha\^{\i}ne de caract\`eres contenant
les cinq premiers caract\`eres du nom du fichier. Le reste du nom est
compos\'e par le syst\`eme en vue d'\'eviter d'obtenir deux noms identiques.
          \item[Sortie sur imprimante : ] Nom d'imprimante suivi du
caract\`ere << : >> .\\
Exemple  : \tt ricoh2:\rm\\
 \em L'imprimante doit \^etre une
imprimante PostScript.\em
       \end{description}
\end{argdesc}


\section{Modules de gestion du terminal alphanum\'erique}
%----------------------------------------------------------

   Ces modules fonctionnent en environnement VTxxx (FALCO) ou << xterm >>.

\subsection{Init. du terminal : \tt INITRM}
\fsub{INITRM}

Cette routine initialise le terminal en mesurant son nombre de lignes et en
r\'eservant au bas de l'\'ecran une ligne d'informations utilisable par le
programmeur pour donner \`a l'utilisateur des renseignements sur le
programme en cours. Cette ligne s'affiche en vid\'eo inverse.

\em Il est fortement recommand\'e d'appeler syst\'ematiquement cette routine
en d\'ebut de programme.\em

\subsection{Lecture du nombre de lignes du terminal : \tt GETPAG}
\fsub{GETPAG(NLI)}

Cette routine rend dans son argument entier \tt NLI\rm\ le nombre de lignes
du terminal (non compris la ligne d'information).

\subsection{Ecriture de la ligne d'informations : \tt WRSTAT}
\fsub{WRSTAT(TEXTE)}

Cette routine \'ecrit la cha\^{\i}ne de caract\`eres contenue dans la
variable \tt TEXTE\rm\ de type \tt CHA\-RAC\-TER\rm\ (80 caract\`eres maximum)
sur la ligne d'informations. Si la longueur du texte est inf\'erieure \`a 80
caract\`eres, le texte est centr\'e automatiquement.

\subsection{Init. des cl\'es PF : \tt INPFKY}
\fsub{INPFKY(NKEY,TEXTE)}

Cette routine permet de programmer les cl\'es \framebox{PF1} \ldots
\framebox{PF4} du clavier, en associant \`a une cl\'e un texte qui sera
envoy\'e par la frappe de cette cl\'e lors d'une entr\'ee de donn\'ees par
la routine \tt RPSLDS\rm.
\begin{argdesc}
    \darg{NKEY :} Entier donnant le num\'ero de cl\'e (1 \`a 4).
    \darg{TEXTE :} Cha\^{\i}ne contenant le texte associ\'e \`a la cl\'e (80
                    caract\`eres maximum).
\end{argdesc}

\noindent {\bf Remarques :}
\begin{itemize}
    \item Les cl\'es PF ne sont utilisables qu'apr\`es les avoir arm\'ees par
la routine \tt ENPFKY\rm.
    \item Les cl\'es PF sont automatiquement d\'esarm\'ees apr\`es chaque
entr\'ee par \tt RPSLDS\rm, \em ce qui efface \'egalement le contenu de la
ligne d'informations\em.
    \item Les cha\^{\i}nes de caract\`eres associ\'ees par d\'efaut aux cl\'es PF
sont :  PF1, PF2, PF3, PF4.
\end{itemize}

\subsection{Armement des cl\'es PF : \tt ENPFKY}
\fsub{ENPFKY(INFO)}

Cette routine arme les cl\'es PF, et \'ecrit le texte contenu dans la
cha\^{\i}ne
\tt INFO\rm\ sur la ligne d'in\-for\-ma\-tions. Ce texte est sens\'e
indiquer \`a l'utilisateur la fonction des cl\'es.
 
\subsection{Positionnement du curseur : \tt SCRLOC}
\fsub{SCRLOC(IL,IC)}

    Positionnement du curseur.
\begin{argdesc}
    \darg{IL :}   Entier ;  num\'ero de ligne.
    \darg{IC :}   Entier ;  num\'ero de colonne.
\end{argdesc}

\subsection{Effacement de l'\'ecran : \tt SCRCLR}
\fsub{SCRCLR(IL,IC)}

    Effet identique \`a \tt SCRLOC\rm\ avec,  en plus effacement de  l'\'ecran 
depuis  la  position demand\'ee jusqu'au bas  de  l'\'ecran.

\section{Entr\'ee de donn\'ees avec \'edition ;  modules RPSLD.}
%---------------------------------------------------------------

   Ces modules fonctionnent en environnement VTxxx (FALCO) ou << xterm >>.

    Les modules RPSLD assurent les entr\'ees de donn\'ees en format libre
sur les consoles, avec possibilit\'e de relire une ligne pr\'ec\'edemment
\'ecrite apr\`es l'avoir \'eventuellement modifi\'ee au clavier.  

Pour utiliser les modules RPSLD il faut tenir compte des remarques suivantes
: 
\begin{itemize}
  \item Tout  programme ou sous programme utilisant des modules  RPSLD 
    devra comporter les d\'e\-cla\-rations :
 
    \begin{center}\begin{tabular}{l}
                   \verb+CHARACTER*81 TAMP+\\
                   \verb+COMMON/EDIT/TAMP+
    \end{tabular}\end{center}

  \item Toute op\'eration d'entr\'ee de donn\'ees doit d\'ebuter par un
    appel \`a l'une des routine \tt RPSLDS\rm\ ou \tt RPSLDH\rm\ qui
    g\`erent l'\'edition et la lecture physique de la ligne o\`u se trouvent
    la ou les donn\'ees \`a entrer.  Ensuite, dans le cas o\'u \tt
    RPSLDS\rm\ a \'et\'e utilis\'ee, et pour chaque donn\'ee \`a entrer
    depuis la ligne \'edit\'ee, il faut appeler le programme de conversion
    appropri\'e.

  \item Les routines de conversion ne  comportent qu'un seul argument.

  \item Dans le cas o\`u la lecture par \tt RPSLDS\rm\ est pr\'ec\'ed\'ee
    d'une \'ecriture (sur la m\^eme ligne), cette derni\`ere doit se faire
    dans le tampon \tt
TAMP\rm\ ; ensuite il faut appeler \tt RPSLDS\rm, puis faire les
    op\'erations de lecture.

       \pagebreak 
              Exemple : 

    \begin{verbatim}
        WRITE(TAMP,100)
    100 FORMAT(' ENTREZ DEUX ENTIERS ET UN FLOTTANT :  1 23 1.23')
        CALL RPSLDS(0)
        CALL RPSLDI(I)
        CALL RPSLDI(J)
        CALL RPSLDE(X)
    \end{verbatim}

  \item Dans le cas o\`u l'entr\'ee par \tt RPSLDS\rm\ doit se faire sur une
    ligne vierge, il suffit d'initialiser TAMP \`a blanc : 

    \begin{verbatim}
        TAMP=' '
        CALL RPSLDS(1)
        CALL RPSLDI(I)
    \end{verbatim}

  \item Les entr\'ees successives sont sauvegard\'ees dans une pile
   circulaire de vingt lignes de sorte qu'il est possible, lors d'une
   entr\'ee de donn\'ees, de rappeler les commandes pr\'ec\'edentes en
   utilisant \em les curseurs verticaux\em\ (comme sur le VAX). Il est
   \'egalement possible d'injecter directement par programme (nouvelle
   routine \tt RPSLDP\rm) des lignes de texte dans la pile en vue d'un
   rappel ult\'erieur.

  \noindent {\bf Remarque.} Si l'on effectue une entr\'ee apr\`es \em avoir
d\'eplac\'e le pointeur de pile avec les curseurs\em, celui-ci se
place sur la position imm\'ediatement suivante dans la pile
(comportement diff\'erent du VAX), permettant ainsi d'envoyer facilement des
donn\'ees pr\'ea\-lablement empil\'ees en pressant alternativement les touches
\framebox{$\downarrow$} et \framebox{Return}.

  \item Lors de l'\'edition d'une ligne, outre l'utilisation des curseurs
verticaux pour le rappel des donn\'ees de la pile, plusieurs autres
facilit\'es ont \'et\'e ajout\'ees, pour se d\'eplacer plus rapidement dans
la ligne \`a \'editer, g\'en\'eralement constitu\'ee de plusieurs champs
s\'epar\'es par des espaces :

       \begin{itemize}
            \item Initialement, le curseur est plac\'e au d\'ebut du premier
champ.
            \item La touche \framebox{Tab} avance au d\'ebut du champ
suivant.
            \item L'appui simultan\'e des touches \framebox{Ctrl} et
\framebox{H} fait passer au d\'ebut du premier champ.
            \item L'appui simultan\'e des touches \framebox{Ctrl} et
\framebox{E} fait passer au d\'ebut du dernier champ.
            \item La touche \framebox{Del} efface le caract\`ere plac\'e
avant le curseur.
            \item L'appui simultan\'e des touches \framebox{Ctrl} et
\framebox{A} fait basculer entre le mode << surimpression >> (par d\'efaut en
d\'ebut d'\'edition) et le mode << insertion >>.
       \end{itemize}

\end{itemize}

    Les sous-programmes (de type \tt SUBROUTINE\rm) impl\'ement\'es sont
d\'ecrits dans les sections suivantes.

\subsection{Entr\'ee de ligne avec plusieurs champs \'editables : \tt RPSLDS}
\fsub{RPSLDS(N)}

                   Ecrit le tampon TAMP sur l'\'ecran, puis
		   g\`ere l'\'edition de la ligne par le clavier.  Si
		   l'argument entier \tt N\rm\ n'est pas nul, l'\'edition
		   s'effectue \`a partir du \tt N\rm i\`eme caract\`ere ; si
		   \tt N\rm\ = 0, l'\'edition s'effectue apr\`es le
		   caract\`ere << : >> s'il est trouv\'e, sinon \`a partir du
		   d\'ebut de la ligne.

   {\bf Remarque.} Le premier caract\`ere du tampon TAMP est trait\'e \`a
l'impression comme un caract\`ere de format. Le caract\`ere << 1 >> provoque
un effacement de l'\'ecran, << 0 >> un interligne suppl\'ementaire, et tout
autre code, une \'ecriture normale. \em Ce traitement est effectu\'e
directement par la routine, sans utiliser la sortie FORTRAN standard\em\ ; en
effet les nouveaux compilateurs FORTRAN n'effectuent plus ce traitement. Il
n'est donc plus n\'ecessaire, sur les machines SUN, d'ins\'erer en d\'ebut
de programme l'instruction (non por\-ta\-ble sur d'au\-tres sys\-t\`e\-mes) :

   \begin{center}
                   \tt OPEN(6,FORM = 'p')\rm
   \end{center}

\subsection{Entr\'ee de lignes dans la pile : \tt RPSLDP}
\fsub{   RPSLDP(N)}

Cette routine effectue un traitement initial identique \`a \tt RPSLDS\rm\ ;
le texte est imprim\'e, mais \em aucune entr\'ee au clavier n'est
effectu\'ee\em. Au lieu de cel\`a, la partie \'editable du texte est ajout\'ee
dans la pile pour pouvoir \^etre rappel\'ee lors d'une entr\'ee ult\'erieure
par action sur les curseurs verticaux.

\subsection{Entr\'ee d'une ligne compl\`ete : \tt RPSLDH}
\fsub{  RPSLDH(ITAB)}
                   Lecture de la partie \'editable de la
		   ligne (80 caract\`eres maximum) dans un tableau entier de
		   type \tt INTEGER~ITAB(20)\rm\ ou une cha\^{\i}ne \tt
                   CHARACTER*80~ITAB\rm .

\subsection{Entr\'ee de donn\'ees situ\'ees dans un tableau : \tt RPSDEC}
\fsub{RPSDEC(ARG)}
                   Si \tt ARG\rm\ est le nom d'un tableau de 20
		   entiers ou d'une cha\^{\i}ne de 80 caract\`eres, alors les
		   lectures ult\'erieures s'effectueront dans ce tableau.
		   Utilisable pour analyser une ligne entr\'ee par \tt
                   RPSLDH\rm.
		   Si \tt ARG\rm\ = 0 , retour \`a la lecture normale au clavier.

\subsection{Blocage du clavier en majuscules : \tt RPSUPP}
\fsub{   RPSUPP(CAPS)}

Si la valeur de l'argument \tt CAPS\rm\ de type \tt LOGICAL\rm\ est \tt
.TRUE.\rm, tous les caract\`eres entr\'es par \tt RPSLDS\rm\ sont convertis
en majuscules et imprim\'es comme tels sur l'\'ecran, quel que soit l'\'etat
du clavier. Un appel avec la valeur \tt .FALSE.\rm\ restitue le
fonctionnement normal.

\em L'usage de cette routine est recommand\'e dans les programmes
conversationnels ; il \'evite d'avoir \`a tester les majuscules et les
minuscules dans le programme, et minimise les erreurs de l'op\'erateur.\em

\subsection{Routines de conversion : \tt RPSLDx}

Ces routines sont utilis\'ees pour lire des valeurs de diverses natures
situ\'ees dans les champs successifs d'une ligne entr\'ee p\'ealablement par
\tt RPSLDS\rm, ou point\'ee par \tt RPSDEC\rm.

\begin{argdesc}
\darg{RPSLDE(X) :}
                   Lecture d'un flottant.

\darg{RPSLDD(D) :}
                   Lecture d'un flottant double pr\'ecision.

\darg{RPSLDI(I) :}
                   Lecture   d'un  entier  normal  (32  bits).

\darg{RPSLDM(V) :}
                   Lecture  d'un  mot  de  4  caract\`eres  dans  un 
                   flottant.  Attention\rm :   la  valeur de ce mot  ne 
                   peut  \^etre compar\'ee qu'avec une autre  variable 
                   flottante initialis\'ee par \tt DATA\rm, et non avec une 
                   cha\^{\i}ne de caract\`eres.

\darg{RPSLDL(W) :}
                   Lecture  d'un  mot  de  8  caract\`eres  dans  un 
                   flottant double pr\'ecision. M\^eme remarque.

\darg{RPSLDX(X) :}
                   Si le premier caract\`ere lu est num\'erique (ou un 
                   signe ou un point),  on lit un nombre flottant, 
                   sinon on lit un mot de 4 caract\`eres comme  pour 
                   \tt RPSLDM\rm.

\darg{RPSLDC(C) :}
                   Lecture  d'un complexe (partie  r\'eelle,  partie 
                   imaginaire).

\darg{RPSLDZ(I) :}
                   Lecture d'un entier en hexad\'ecimal.

\darg{RPSLDT(T) :} Lecture d'une cha\^{\i}ne (ou sous cha\^{\i}ne) de
		   caract\`eres ne comportant pas d'es\-pa\-ce.
\end{argdesc}

\subsection{Contr\^ole des erreurs : \tt IRPSLD}
\fsub{   IRPSLD(IER)} 
   Routine de contr\^ole des erreurs. L'appel \`a cette routine rend dans la
variable enti\`ere \tt IER\rm\ un code d'erreur portant sur la derni\`ere
op\'eration de conversion, et arme le contr\^ole des erreurs. L'absence
d'erreur est indiqu\'ee par \tt IER\rm\ = 0 .

   Le contr\^ole des erreurs est d\'esarm\'e par chaque appel \`a \tt
RPSLDS\rm. C'est pourquoi \tt IRPSLD\rm\ \em doit \^etre invoqu\'e apr\`es
appel \`a \em
\tt RPSLDS\rm\ \em et avant chaque routine de conversion\em.

  Il peut se produire deux sortes d'erreurs d'entr\'ee de donn\'ees:

  \begin{itemize}
      \item Les donn\'ees sont syntaxiquement erron\'ees. Si le contr\^ole
d'erreur n'est pas arm\'e, le programme est << abort\'e >> avec un message
d'erreur. Sinon \tt IER\rm\ est strictement positif et contient un code qui
peut d\'ependre de la machine utilis\'ee.

      \item Il manque une ou plusieurs donn\'ees dans la ligne lue. Cette
erreur n'est pas fatale, m\^eme en absence de contr\^ole, et la variable \tt
IER\rm\ contient un nombre n\'egatif dont la valeur absolue est \'egale au
nombre de donn\'ees manquantes.
   \end{itemize}

   En cas d'erreur, les donn\'ees lues par les programmes de conversion sont
ainsi trait\'ees : s'il s'agit de variables num\'eriques, elles conservent
leur pr\'ec\'edente valeur ; les variables litt\'erales sont remplies par
des espaces.

\subsection{Type des donn\'ees lues par \tt RPSLDX\rm\ : \tt JRPSLD}
\fsub{JRPSLD(ITYPE)}
                    Apr\`es un appel \`a \tt RPSLDX\rm, cette
		   routine rend dans l'entier \tt ITYPE\rm\ la nature de
		   l'argument lu :

                   \begin{center}\begin{tabular}{ll} 
                        1 &    flottant\\
                        2 &    mot
                   \end{tabular}\end{center}

\section{Fonctions associ\'ees \`a des menus}
%--------------------------------------------

    Deux fonctions ont \'egalement \'et\'es \'ecrites pour faciliter les
dialogues au moyen de menus.

Ces fonctions posent une question et traitent les diff\'erentes r\'eponses
possibles.

\subsection{Entr\'ee d'options d\'efinies dans une cha\^{\i}ne de
             caract\`eres : \tt IDLG}
\ffun{IDLG(FORME,CTABREP,NREP)} 

\begin{argdesc} 

\darg{FORME :} Deux possibilit\'es :
    \begin{itemize}
          \item    cha\^{\i}ne de caract\`eres contenant le
		   format d'impression de la question ; le texte \`a
		   imprimer se termine par << : >>.  
          \item    Nombre 0, ou cha\^{\i}ne de caract\`eres commen\c{c}ant
                   par \tt CHAR(0)\rm. Dans ce cas, la question est prise
                   dans le tampon \tt TAMP\rm\ du \tt COMMON EDIT\rm.
    \end{itemize}

   Dans les deux cas, la question sera exploit\'ee par \tt RPSLDS\rm\ et le
   premier caract\`ere de celle-ci sera donc trait\'e en temps que
   caract\`ere de format (Cf. \tt RPSLDS\rm).
\darg{CTABREP :}
		   Cha\^{\i}ne de caract\`eres contenant la succession des
		   r\'eponses possibles (chaque r\'eponse sur 4
		   caract\`eres, cadrage gauche).  
\darg{NREP :} Nombre de r\'eponses possibles.  En sortie, la fonction rend le
		   num\'ero de la r\'eponse.  

       \pagebreak
                   Exemple :

                   \begin{center}\begin{tabular}{l}
                       \verb+C-----AFFICHAGE DU MENU-----+\\
                       \verb+      WRITE(6,100)+\\
                       \verb+100   FORMAT('CONTINUER : C'/+\\
                       \verb+     *       'ARRETER   : A')+\\
                       \verb+C-----ENTREE DE LA REPONSE+\\
                       \verb+      I=IDLG('('' CHOIX : '')','C   A   ',2)+\\
                       \verb+      GOTO (1,2), I+
                   \end{tabular}\end{center}

                    L'entr\'ee de la r\'eponse de l'exemple pr\'ec\'edent
                    peut \'egalement s'\'ecrire:
                   \begin{center}\begin{tabular}{l}
                       \verb+C-----ENTREE DE LA REPONSE+\\
                       \verb+      TAMP=' CHOIX : C '+\\
                       \verb+      I=IDLG(0,'C   A   ',2)+\\
                       \verb+      GOTO (1,2), I+
                   \end{tabular}\end{center}

                    Remarquons la r\'eponse << C >> propos\'ee par d\'efaut
                    dans ce dernier cas.

\end{argdesc}              

\subsection{Entr\'ee d'options d\'efinies dans un tableau : \tt IDLGA}
\ffun{IDLGA(FORME,TABREP,NREP)} 

   Cette routine est analogue \`a la pr\'ec\'edente, mais l'argument \tt
TABREP\rm\ d\'esigne ici un tableau de \tt CHARACTER*4\rm\ dimensionn\'e \`a
\tt NREP\rm, contenant les r\'eponses possibles.

\section{Lecture de donn\'ees sur un fichier.}
%----------------------------------------------

   Les modules RPSLD sont maintenant capables de traiter des donn\'ees lues
sur un fichier. Il suffit pour cel\`a de rediriger sur ce fichier l'entr\'ee
standard du programme, lors du lancement de celui-ci : 

\begin{center}
     {\tt programme < fichier\_de\_donn\'ees}
\end{center}

   Le fichier de donn\'ees comportera une ligne par appel \`a \tt RPSLDS
\rm\ ou \tt RPSLDH\rm. Afin de faciliter la relecture du fichier de
donn\'ees par le programmeur, ce fichier pourra comporter des lignes de
commentaires \em commen\c{c}ant par le caract\`ere\em\ << \# >>. Ces lignes
seront ignor\'ees lors de l'ex\'ecution du programme.

\section{Utilisation des biblioth\`eques.}
%------------------------------------------

    Pour pouvoir ex\'ecuter un programme graphique, il est imp\'eratif 
que le fichier \tt .login\rm\  de l'uti\-lisa\-teur comporte les commandes : 
  \begin{center}\begin{tabular}{l}
     \verb+/home/etc/rpsld_login+\\
     \verb+/home/etc/fminigraf_login+
  \end{tabular}\end{center}

    L'ex\'ecution de ces commandes initialise les variables
d'en\-viron\-ne\-ment : 

    \begin{center}\begin{tabular}{l}
    \verb/LNK_RPSLD/\\
    \verb/LNK_FMINIGRAF/ 
    \end{tabular}\end{center}

\noindent
contenant les options n\'ecessaires \`a l'\'edition de lien d'un programme.

    Si le fichier \tt.login \rm n'a pas \'et\'e ex\'ecut\'e, on pourra le
lancer manuellement par:

     \begin{center}
     {\tt source ~/.login}
     \end{center}

    Pour l'\'edition de liens d'un programme, on utilisera \tt f77\rm\ sur
SUN et \tt fort77\rm\ sur HP. Si on utilise une commande directe on ajoutera
les options:

      \begin{center}
      \verb/$LNK_RPSLD $LNK_FMINIGRAF/
      \end{center}

\noindent
Et dans un \tt Makefile\rm :

      \begin{center}
      \verb/$(LNK_RPSLD) $(LNK_FMINIGRAF)/
      \end{center}

   Les routines graphiques se trouvent dans la biblioth\`eque \tt
libfminigraf\rm,
et \em toutes les autres routines sont dans \em \tt librpsld\rm.

\noindent
   {\bf Remarque.} \em Si la biblioth\`eque \em {\tt librpsld}\em\ est utilis\'ee
seule\em\ il faut ajouter au programme un module << bidon  >> :

   \begin{center}\begin{tabular}{l}
      \verb/SUBROUTINE FBGTXT/\\
      \verb/RETURN/\\
      \verb/END/
   \end{tabular}\end{center}

\end{document}
