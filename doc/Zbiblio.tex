\begin{thebibliography}{99}  %%References 
\addcontentsline{toc}{section}{\numberline{}REFERENCES}

\bibitem{Biblio1} F.~M\'eot et S.~Val\'ero, \textsl{Manuel
d'utilisation de  Zgoubi}, report IRF/LNS/88-13, CEA Saclay, 1988.  

\bibitem{Biblio2} 
F.~M\'eot and S.~Val\'ero, \textsl{Zgoubi users' guide}, 
Int.~Rep. CEA/DSM/LNS/GT/90-05, CEA Saclay (1990) \& TRIUMF report TRI/CD/90-02 (1990). 

\bibitem{Biblio2b} F.~M\'eot and S.~Val\'ero, \textsl{Zgoubi users' guide - Version 3}, 
Int.~Rep. DSM/LNS/GT/93-12, CEA Saclay (1993). 

\bibitem{Biblio2c} 
F.~M\'eot and S. Valero,  
\textsl{Zgoubi users' guide - Version 4}, 
FNAL Tech.~Rep. FERMILAB-TM-2010 (Aug.~1997),  \& 
Int.~Rep. CEA~DSM~DAPNIA/SEA-97-13, Saclay (Oct.~1997). 

\bibitem{SFZ} http://zgoubi.sourceforge.net/ 

\bibitem{Biblio6} F.~M\'eot, \textsl{The electrification of Zgoubi}, 
Saturne report DSM/LNS/GT/93-09, CEA Saclay (1993) ; F.~M\'eot, \textsl{
Generalization of the Zgoubi method for ray-tracing to include electric 
fields}, NIM A 340 (1994) 594-604. 

\bibitem{Biblio3} D.~Carvounas, \textsl{Suivi num\'erique de particules charg\'ees 
dans un sol\'eno\"\i de}, report de stage, CEA/LNS/GT-1991. 

\bibitem{Biblio4} F.~M\'eot, \textsl{Raytracing in 3-D field maps with Zgoubi}, 
report  DSM/LNS/GT/90-01, CEA Saclay, 1990.

\bibitem{Biblio5} G.~Leleux, \textsl{Compl\'ements sur la physique des acc\'el\'erateurs}, 
cours de DEA Univ.~Paris-VI,  report  IRF/LNS/86-101, CEA Saclay, March 1986. 

\bibitem{Biblio7} F.~M\'eot, \textsl{A numerical method for combined spin tracking and 
raytracing of charged particles}, 
NIM \textbf{A313} (1992) 492, and proc. EPAC (1992) p.747. 

\bibitem{Biblio8} V. Bargmann, L. Michel, V.L. Telegdi,
\textsl{Precession of the polarization of particles moving in a homogeneous electromagnetic field}, 
Phys. Rev. Lett. 2 (1959) 435.
%%%V.~Bargmann \textsl{et al.}, Phys.\ Rev.\ Lett. \textbf{2} (1959) 435.

\bibitem{FMSEA-00-01} F.~M\'eot and J.~Payet, 
\textsl{Numerical tools for the simulation of synchrotron radiation  
loss and induced dynamical effects in high energy transport lines},  
Report DSM/DAPNIA/SEA-00-01, CEA Saclay (2000). 

\bibitem{FMSL/94-22} F.~M\'eot, \textsl{Synchrotron radiation interferences at the LEP miniwiggler}, 
Report CERN SL/94-22 (AP), 1994. 

\bibitem{FMLPYellow} L. Ponce,  R. Jung, F. M\'eot, 
\textsl{LHC proton beam diagnostics using synchrotron radiation},  
Yellow Report CERN-2004-007. 

\bibitem{JDJ} J.D.~Jackson, \textsl{Classical electrodynamics}, 2nd 
Ed., J.~Wiley and Sons, New York, 1975.

\bibitem{after} F.~M\'eot, 
\textsl{A theory of low frequency far-field synchrotron radiation}, 
 Particle Accelerators Vol~62, pp.~215-239  (1999). 

\bibitem{Biblio9} B.~Mayer, personal communication,
CEA Saclay, 1990. 

\bibitem{NedlerMead} Installed by J.~S.~Berg, BNL (2007). \emph{Cf.} \textsl{Detection and remediation of stagnation in the 
Nelder-Mead algorithm using a sufficient decrease condition}, 
  C.~T.~Kelley, Siam J. Optim., Vol. 10, No. 1, pp. 43-55. 

\bibitem{Biblio10} L.~Farvacque \textsl{et al.}, \textsl{Beta user's guide}, Note 
ESRF-COMP-87-01, 1987; 
J.~Payet, IRF/LNS, CEA Saclay, private communication; see also J.M.~Lagniel, 
\textsl{Recherche d'un optimum}, Note IRF/LNS/SM 87/48, CEA Saclay 1987.

\bibitem{Biblio11} F.~M\'eot and N.~Willis, \textsl{Raytrace computation with 
Monte Carlo simulation of particle decay}, internal report CEA/LNS/88-18 
CEA Saclay, 1988. 

\bibitem{Biblio12} H.A.~Enge, \textsl{Deflecting magnets}, in \textbf{Focusing of 
Charged Particles}, ed.~A.~Septier, \textbf{Vol.}  \textbf{II}, pp 203-264, Academic
Press Inc., 1967.

\bibitem{Biblio13} P.~Birien et S.~Val\'ero, \textsl{Projet de spectrom\`etre 
magn\'etique \`a haute r\'esolution pour ions lourds}, \textbf{Section IV} 
p.62, Note CEA-N-2215, CEA Saclay, mai 1981.

\bibitem{Biblio14} V.~M.~Kel'man and S.~Ya.~Yavor, Achromatic
quadrupole electron lenses, Soviet Physics - Technical Physics, vol.~6, No~12, June 1962; \\
S.~Ya.~Yavor \textsl{et als.}, \textsl{Achromatic quadrupole lenses}, NIM \textbf{26} (1964) 13-17. 

\bibitem{Biblio15} A.~Septier et J.~van Acker, \textsl{Les lentilles quadrupolaires 
\'electriques}, NIM \textbf{13} (1961) 335-355; Y.~Fujita and H.~Matsuda, 
\textsl{Third order transfer matrices for an electrostatic quadrupole lens,} 
NIM \textbf{123} (1975) 495-504. 

\bibitem{Biblio16} A.~Septier, Cours du DEA de physique des
particules, optique corpusculaire, Universit\'e d'Orsay, 1966-67, pp.~38-39.  


\newpage

\bibitem{Karets} S.~P.~Karetskaya et als., 
\textsl{Mirror-bank energy analyzers, in Advances in electronics and electron physics}, 
Vol. 89, Acad. Press (1994) 391-491. 

\bibitem{reportNIMFFAG} F.~Lemuet, F.~M\'eot, \textsl{Developements in the ray-tracing code Zgoubi for 
6-D multiturn tracking in FFAG rings}, NIM~A \textbf{547} (2005) 638-651. 

\bibitem{reportICFAFFAG} F. M\'eot,  
\textsl{6-D beam dynamics simulations in FFAGs using the ray-tracing code Zgoubi}, 
ICFA Beam Dyn.Newslett.43:44-50 (2007).

\bibitem{reportNIMFFAGSPI} J.~Fourrier, F.~Martinache, F.~M\'eot, J.~Pasternak, 
\textsl{Spiral FFAG lattice design tools,  application to 6-D tracking in a proton-therapy class lattice}, 
NIM~A~589 (2008). 

\bibitem{Pavel} P. Akishin, JINR, Dubna, 1992.

\bibitem{Biblio17} M.W.~Garrett, \textsl{Calculation of
fields  [...] by elliptic integrals}, J.~Appl.~Phys., \textbf{34}, 9, sept.~1963.  

\bibitem{Biblio18} P.F.~Byrd and M.D.~Friedman, \textsl{Handbook of elliptic integrals 
for engineers and scientists,} pp.~282-283, Springer-Verlag, Berlin, 1954.  

\bibitem{Biblio19} A.~Tkatchenko, \textsl{Computer program UNIPOT}, SATURNE, CEA Saclay, 1982. 

\bibitem{BiblioPlot} J.L.~Chuma, \textsl{PLOTDATA}, TRIUMF Design Note TRI-CO-87-03a.

\bibitem{HAMEL} J.L.~Hamel, \textsl{mini graphic library LIBMINIGRAF}, CEA-DSM, Saclay, 1996.

\bibitem{Grorud} E.~Grorud, J.L.~Laclare, G.~Leleux, 
\textsl{R\'esonances de d\'epolarisation dans SATURNE~2}, 
Int. report GOC-GERMA 75-48/TP-28, CEA Saclay (1975), 
and, Home Computer Codes POLAR and POPOL, IRF/LNS/GT, CEA Saclay (1975).

\bibitem{Froissart} M.~Froissart et R.~Stora, 
\textsl{D\'epolarisation d'un faisceau de protons polaris\'es dans un synchrotron},
 NIM 7 (1960) 297-305.

\bibitem{Spes2} J.~Thirion et P.~Birien, 
\textsl{Le spectrom\`etre II, Internal Report DPh-N/ME}, 
CEA Saclay, 23 D\'ec. 1975; \\
H.~Catz, \textsl{Le spectrom\`etre SPES~II}, Internal Report DPh-N/ME, CEA Saclay, 1980. 

\bibitem{BNL}  P.~Pile, I-H.~Chiang, K.~K.~Li, C.~J.~Kost, J.~Doornbos, F.~M\'eot et als., 
\textsl{A two-stage separated 800-MeV/c Kaon beamline}, 
 TRIUMF and BNL Preprint (1997).

\bibitem{ZgCern} F.~M\'eot, 
\textsl{The raytracing code Zgoubi}, CERN SL/94-82 (AP) (1994), 
3rd Intern. Workshop on Optimization and Inverse Problems 
in Electromagnetism, CERN, Geneva, Switzerland, 19-21 Sept. 1994. 

\bibitem{NumRec} W.~H.~Press et {\it als.}, \textsl{Numerical recipes}, Cambridge Univ. Press (1987).  

\bibitem{VOKostroun} V.~O.~Kostroun, 
\textsl{Simple numerical evaluation of modified Bessel functions and integrals [...]}, 
NIM 172 (1980) 371-374. 

\end{thebibliography}

