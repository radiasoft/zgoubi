\documentclass[12pt]{article}
%\usepackage{draftcopy}
%\usepackage[draft]{graphicx}
\usepackage{graphicx}
\usepackage{amssymb}
\usepackage{lscape}
\usepackage{wrapfig}
\usepackage{times}
\usepackage{color}

\oddsidemargin= -.6in
\evensidemargin=-.6in
\textwidth=10.6in
\topmargin=-0.9in
\footskip=1.in
  
\newcommand{\Br}{\ensuremath{B\!\rho}}
\newcommand{\bull}{\ensuremath{\bullet~}}
\newcommand{\Bz}{\ensuremath{{B_z}}}
\newcommand{\com}{\ensuremath{\it com}}
\newcommand{\dip}{{\it DIPOLE}}
\newcommand{\EFB}{\ensuremath{E\!F\!B}}
\newcommand{\EFBs}{\ensuremath{E\!F\!B\!s}}
\newcommand{\ffag}{{\it FFAG}}
\newcommand{\hbrk}{\hfill \break}
\newcommand{\lab}{\ensuremath{lab}}
\newcommand{\MC}{Monte~Carlo}
\newcommand{\nind}{\noindent}

\newcommand{\blue}{\color{blue}}
\newcommand{\green}{\color{green}}
\newcommand{\red}{\color{red}}

 
\pagestyle{myheadings} 

\markboth{\large \red Playing with Particle Accelerator Physics, on computer} 
         {\large \red Playing with Particle Accelerator Physics, on computer} 



\begin{document}


\sffamily


{
 ~~~~~~~ \hfill \large   \bf F.~M\'eot 

~~~~~~~\hfill \large   \bf CEA \& IN2P3, LPSC
}

~~~~~~~~~~~~~~~~~~~~~~~~~

\vspace{-10mm}

\begin{center} 
{\Huge  
\bf  An introduction to particle accelerators
}
\end{center} 

\tableofcontents





%%%%%%%%%%%%%%%%%%%%%%%%%%%%
\clearpage

\section{\LARGE Generalities }

\begin{itemize}
  \item[$\bullet$] Particle accelerators were born in the quest of  ``atom smashers'', in a context of 
needs for higher and higher  energies, 
beyond natural radioactivity bodies, in the several MeV range~: 

for producing high energy $e-$ and ion beams,  probing the atomic nucleus, 
 creating  new elements and  isotopes 
% energy levels of the nucleus, by measuring E of products. Properties of excited states by measuring angular distributions 
% and other data
% Radio-isotopes : tracers in chemistry, biology, medecine etc.
% Essential data for developing nuclear reactors and weapons

~

For reference~: high energy alpha from radioactive \\ 
particles were $\sim$10~MeV. 

~

  \item[$\bullet$] In the era of nuclear R\&D, civil and military, they \\
 allow(ed) extensive production of data on radio- \\
isotopes, production cross-sections...
  \item[$\bullet$] Very high energies have opened the field of \\
accelerator based particle physics
\hspace{.18\linewidth} \raisebox{-70mm}[0mm][0mm]{\hspace{0mm}
%\includegraphics*[width=10.500cm]{accelerators-Energies.eps}
\includegraphics*[width=13.cm]{livingstonChart_Bryant.eps}
}

  \item[$\bullet$] Energies have increased exponentially over the \\
 years, more or less saturating depending \\
on the technology

  \item[$\bullet$] Later, with discoveries as synchrotron radiation,  \\
hadron-therapy, and given their potential for number \\
of applications, accelerators found themselves  \\
predilection tools in many domains of science~: \\
production of X-rays, medical, industry...
\end{itemize}






%%%%%%%%%%%%%%%%%%%%%%%%%%%%
\clearpage

~

\vspace{-100mm}

\begin{table}[t]
\section*{\LARGE Generalities }

\begin{center}
\bf
\large

{\huge The world of accelerators }

~

\begin{tabular}{|l|c||c|l|l|}
\hline
\\
                    &  \multicolumn{2}{|c|}{Kinetic energy } &  \multicolumn{2}{|c|}{ } \\
\\
\hline
 & & & & \\
                                &  Electrons         &  protons and ions & What for                       &  Records \\
 & & & & \\[1ex]
\hline
Cockcroft-Walton&                    &      1 MeV      & Material science, injector &\\[.6ex]
Van de Graaff   &                    &      20-35 MeV  &  - id -                    & \\[.6ex]
Betatron                        &    10 - 300 MeV    &                 & X-Ray generator            &  Industry \\[.6ex]
Microtron                       &  20 - 1500 MeV     &                 & Science, industry          &  MAMI, 1.5 GeV\\[.6ex]
Cyclotron                       &                    & 10 - 560 MeV    & Material-, bio-sci., medical     & PSI, 590~MeV/1.2~MW\\[.6ex]
Synchro-cyclotron               &                    & 100 - 750 MeV   &                            &  \\[.6ex]
Synchrotron                     &    1 - 10 GeV      & 1 - 500 GeV     &  acceleration, booster     &  SPS, 450 GeV \\[.6ex]
Storage ring                    &   1 - 8 GeV        &                 &  SR                        & Spring-8 LS, 8 GeV \\[.6ex]
Colliders rings                 &  1 - 200 GeV       & 60 GeV - 14 TeV &  HEP, factories  & LEP~II, 200 GeV / LHC, 14~ TeV \\[.6ex]
e-Linacs                        &  few MeVs             &                 &  Industry, medic., science    & \\[.6ex]
Linacs                          &   20 MeV - 50 GeV  & 50 - 800 MeV    &  HEP, X-FEL             & SLAC, 50 GeV / LAMPF, 800 MeV  \\[.6ex]
Linear collider                 &  50 GeV - 1 TeV    &                 & HEP - projects          & SLC, 100 GeV - CLIC, 1~TeV \\[.6ex]
HPPA, synchrotron      &                    &         1 GeV   & Material science, $\mu$, $n$, $\nu$,  ADS   & ISIS, 800~MeV/160~kW     \\[.6ex]
HPPA, linac            &                    &    1 GeV        & idem   & SNS, 1~GeV/1.4~MW     \\[.6ex]
\hline
\end{tabular}
\end{center}

\end{table}





%%%%%%%%%%%%%%%%%%%%%%%%%%%%
\clearpage

\section*{\LARGE Generalities }

\begin{itemize}
  \item[$\bullet$]  We will follow a common classification that distinguishes four major concepts 
in the develoment of accelerators~: 
  \begin{itemize}
    \item[-] Electrostatic accelerators 
    \item[-] Resonant acceleration
    \item[-] Phase stability
    \item[-] Strong focusing
  \end{itemize}
\nind   However, from the beginning, and this is still true nowadays, 
researches were held in several directions, in parallel~: 
  \begin{itemize}
    \item[-] 1924, Gustav~Ising, tentative resonant linac acceleration, failed on operating
    \item[-] 1928, Wider\"{o}e, first linac ever, resonant  acceleration,  and tentative betatron
    \item[-] 1932,  John Douglas Cockcroft and Ernest Thomas Sinton Walton, 
Cavendish Lab., pushed by Rutherford, voltage multiplier, first transmutation with 700~keV proton beam
    \item[-] 1930, Ernest~O.~Lawrence invents the cyclotron 
    \item[-] 1929 - 1931, ``Van de Graaff'' electrostatic generator model, two 
spherical bulbes $\Phi$60~cm, reached 1.5~MV
    \item[-] 1932, E.O.~Lawrence \& M.S.~Livingston, Univ. of California, operation of the first cyclotron, 1.25~MeV protons. 
Nuclear reactions just a few weeks after John~Cockcroft and Ernest~Walton
    \item[-] Etc... this is the goal of this lecture to give an overview of all this and the rest...
  \end{itemize}
\end{itemize}


\end{document}


