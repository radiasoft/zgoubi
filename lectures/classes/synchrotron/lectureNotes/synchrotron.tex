
%Superscripts and subscripts that are words or abbreviations, as in
%\( \pi_{\mathrm{low}} \), should be typed as roman letters; this is
%done as \verb|\( \pi_{\mathrm{low}} \)| instead of \( \pi_{low} \)
%done by \verb|\( \pi_{low} \)|.

%User-defined macros should be placed in the preamble of the chapter,
%and not at any other place in the document. Definitions made using
%the commands \verb|\newcommand,| \verb|\renewcommand,|
%\verb|\newenvironment| or \verb|\renewenvironment| should be used


\chapter[Synchrotron]{Synchrotron}\label{chapSynchrotron}


\section{Introduction \label{secSynchIntro}}




\section{Constituents \label{secSynchConstituents}}


The basic optical constituents of a synchrotron, and sufficient to implement this circular accelerator method, 
are in the number of four~:  field-free space, dipole, quadrupole and an accelerating system.  


\section{A synchrotron ring \label{secSynchRing}}


\section{Acceleration, synchrotron motion \label{secSynchAccel}}



\section{Betatron motion \label{secSynchBetatronMotion}}


\section{Injection \label{secSynchInj}}

We inspect  the particular technique of ``multiturn injection''...
The method is based on the deformation of the periodic orbit in the vicinity of an injection septum...

************ Leleux



\section{Resonant extraction \label{secSynchSlowXtr}}

Resonant extraction finds application in beam extraction at a slow rate, milliseconds up to seconds. 
for use on fixed targets, which includes for instance tumor therapy. 

************ Leleux




\section{Bibliography \label{SecBiblioSynch}}

G.~Leleux, Circular accelerators, http://  ******* mettre e-copy sur un site web, en tant que docu public CEA 


