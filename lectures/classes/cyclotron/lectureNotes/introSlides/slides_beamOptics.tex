\documentclass[12pt]{paper}
%\usepackage{draftcopy}
%\usepackage[draft]{graphicx}
\usepackage{graphicx}
\usepackage{amssymb}
\usepackage{lscape}
\usepackage{wrapfig}
\usepackage{times}
\usepackage{color}

 \oddsidemargin =-0.7in                            
 \evensidemargin=-0.7in                                                          
 \textwidth=7.8in                         
 \textheight=10.in                       
 \topmargin=-.8in                                                                
 \footskip=0in                                                                   
%\oddsidemargin=-.6in
%\evensidemargin=-.6in
%\textwidth=10.in
%\topmargin=-0.8in
%\footskip=0.in
%\oddsidemargin=-.5in
%\evensidemargin=-.5in
%\textwidth=10.in
%\topmargin=-.7in
%\footskip=-5.in
  
\newcommand{\A}{\mathcal{A}}
\newcommand{\B}{\ensuremath{\vec B}}
\newcommand{\Br}{\ensuremath{B\!\rho}}
\newcommand{\Bz}{\ensuremath{{B_z}}}
\newcommand{\com}{\ensuremath{\it com}}
\newcommand{\dip}{{\it DIPOLE}}
\newcommand{\dl}{\ensuremath{\delta}}
\newcommand{\ds}{\ensuremath{\vec {ds}}}
\newcommand{\E}{\ensuremath{\vec E}}
\newcommand{\EFB}{\ensuremath{E\!F\!B}}
\newcommand{\EFBs}{\ensuremath{E\!F\!B\!s}}
\newcommand{\F}{\ensuremath{\vec F}}
\newcommand{\ffag}{{\it FFAG}}
\newcommand{\hbrk}{\hfill \break}
\newcommand{\lab}{\ensuremath{lab}}
\newcommand{\ms}{\medskip}
\newcommand{\MC}{Monte~Carlo}
\newcommand{\nib}{\noindent $\bullet$~}
\newcommand{\nin}{\noindent}
\newcommand{\p}{\ensuremath{\vec p}}
\newcommand{\R}{\ensuremath{\vec R}}
\newcommand{\Res}{\ensuremath{\mathcal{R}}}
\newcommand{\mR}{\ensuremath{\mathcal{R}}}
\newcommand{\mS}{\ensuremath{\mathcal{S}}}
\newcommand{\vv}{\ensuremath{\vec v}}


\newcommand{\black}{\color{black}}
\newcommand{\red}{\color{red}}
\newcommand{\blue}{\color{blue}}
\newcommand{\green}{\color{green}}

 
\pagestyle{myheadings} 

\markboth{\large \color{blue} JUAS 2012, Archamps, 12 Jan. 2012, Beam Optics, F. M\'eot}
         {\large \color{blue} JUAS 2012, Archamps, 12 Jan. 2012, Beam Optics, F. M\'eot}




\begin{document}

\landscape


\sffamily


%\thispagestyle{empty}
  
{
 ~~~~~~~ \hfill \large   \bf F.~M\'eot 

~~~~~~~\hfill \large   \bf BNL, C-AD/AP, Upton, NY
}

\begin{center} 
\Huge  
\bf An introduction to beam optics\footnote{\large \bf After (i) G.Leleux, Accelerateurs Circulaires, Lectures, CEA Saclay (1978), 
(ii) L. Farvacque, A. Tkatchenko, \textsl{in} Ensembles de d\'etection magn\'etique du Laboratoire National SATURNE,  CEA Saclay (1980), (iii) A.~Septier, Charged Particle Optics.}
\end{center} 


{\blue   \Large \bf
\nib\ We will address in this lecture  the theory of the  guiding and focusing of charged particles in accelerator structures. 
We will start discussing the methods of ``Beam Optics'' by  introducing the basic tools  needed in that domain~: 



(i) We will investigate how particle motion in electrostatic fields and magnetostatic fields is governed by the fundamental 
laws of dynamics 



and how  approximations of these into convenient  mathematical  tools will make our lives sometimes simpler 



(ii) We will introduce the basic ``optical elements'' used in accelerator structures as beam lines, circular 
accelerators, spectrometers, etc.,  which ensure guiding, focusing and other beam manipulations. 



\nib\ Then, we will ``visit''~: discuss, understand, some typical examples of such optical assemblies. 
}


\begin{center}
\includegraphics*[width=0.64\linewidth]{./beamOptics_Figs/AGSString.eps}
\end{center}



\clearpage

\tableofcontents




\clearpage

\nib\ In a general manner, the design  

~

- of  beam transport lines, 

~

- and of circular accelerators as well -  including the largest ones !  

~

\nin in first approximation only require elementary  functions as parabola,  sine, cosine,  hyperbola, exponential. 

~

~

\nib\ The complexity of optical assemblies arises from the variety of these laws and of their combination~: 

~

a particle will 
follow arcs of circles, arcs of parabola, sine trajectories, ``pseudo-sine'' laws, etc. 

~

~

\nib\ As a consequence, a very limited mathematical toolbox makes it is possible to deal with 
sometimes very complex optical assemblies. 








\clearpage

\section{ \LARGE Motion of a charged particle in electric or magnetic fields}

\bf \Large 

\nib\ Optical rays are deflected, reflected, using dioptric and/or catadioptric systems, 

~

\nib\ charged particles are 
deflected, reflected,  \textsl{and accelerated too}, 

~

- using   magnetic fields 

~

-  electric fields 

~

- combinations of both, 

~

-  either static or in addition, in some cases, varying in time. 

~

~

\nib\ Prior to looking in a detailed way at the optical elements proper to  charged particle optics, 

~

we will first review 
the basis of the motion of charged particles in  magnetic and electric fields. 



\clearpage 

\section*{ \LARGE Notions of dynamics}

\nib\ The force that acts on a charged particle, 

\hfill is the Lorentz force~:  {\blue  \LARGE \fbox{$  \mathbf{\vec F = q (\vec E + \vec v \times \vec B)}$} } \hfill ~ 

~

$q$ : charge of the particle (Coulomb, C)

~

$\vv$ : velocity of the particle (m/s)

~

$\vec E$~: electric field, in Volt/m (V/m)

~

$\vec B$~: magnetic field, in Tesla (T)


~





\clearpage



{\blue \nib\ The MAGNETIC FORCE~: }
  
~


\nib\ A manifestation of the magnetic force is  the Laplace force on an electrical circuit~: 
$$\vec F = I \, \vec{dl} \times \vec B$$


~

\nib\ Another manifestation is the force experienced by  charged particle with  velocity, $\vv$~: 
$$\F = q \vv \times \B ~ , $$

\nin Under the effect of $\F$ the charged particle undergoes a deviation, its trajectory is curved. 


~

\nib\ A magnetic force does not work~: 

~

$\F  = q (\vv \times \B) $ entails  that $\F $ is orthogonal to $\vv = \ds / dt$, 

as a consequence,  
$$d\mathcal{T} = \F  . \ds = q (\vv \times \B) . \ds \equiv 0  = q (\vv \times \B) . \vv \ dt \equiv 0 $$

~

{\blue An important consequence~:  magnetic forces cannot change particle energy, they can only 
change the direction of the velocity vector, i.e., \textsl{deviate} particles. }






\clearpage 


\underline{Two rules that allow inferring how a moving charged particle is deviated by a magnetic field }

~

Both  rules  yield the orientation of $\F$~: 


{\blue \nib\  $I\vec{dl}$ , $\B$ and $\F$, in that order, form a direct triedra~: }
\raisebox{-10mm}[0mm][0mm]{ ~  ~  ~  ~  \includegraphics*[width=0.33\linewidth]{./beamOptics_Figs/AGSString.eps}}
%\hspace{60mm}
%\raisebox{-10mm}[0mm][0mm]{\includegraphics*[bbllx=0,bblly=0,bburx=300,bbury=254,width=4cm]{./beamOptics_Figs/fFoc.eps}\hspace{-5mm}}
%~     ~ ~ ~ ~ ~ ~ ~ 

~

~

{\centering

\hfill ``Horizontally focusing dipole''   \hfill   ``Horizontally defocusing  dipole'' \hfill ~

\hfill ``Vertically defocusing dipole''   \hfill   ``Vertically focusing  dipole'' \hfill ~

\includegraphics*[width=9cm]{./beamOptics_Figs/fFocWF.eps}\hspace{30mm}
\includegraphics*[width=9cm]{./beamOptics_Figs/dFocWF.eps} 


\begin{minipage}[b]{.4\linewidth}

~

\vspace{30mm}

{\blue \nib\ Rule of the 3 right hand fingers~: }
%\hspace{75mm}
%\raisebox{0mm}[0mm][0mm]{\includegraphics*[bbllx=0,bblly=0,bburx=300,bbury=254,width=4cm]{./beamOptics_Figs/dFoc.eps} }


\end{minipage}
\begin{minipage}[b]{.49\linewidth}
%\hfill ~ 
\raisebox{-30mm}[0mm][0mm]{
\includegraphics*[bbllx=0,bblly=0,bburx=300,bbury=254,width=11cm]{./beamOptics_Figs/3doigts.eps} 
}
%\hfill ~ 

\end{minipage}

}%\centering



\clearpage

\subsection*{\Large \underline{Discussing the fundamental equation of dynamics}}


\begin{minipage}[b]{.49\linewidth}
\centering
\Large

\blue 

\underline{Classical mechanics}

~

$m \frac{ \textstyle{d\vv}}{\textstyle{dt}} = \F $, $m$ is constant

\end{minipage}
\hspace{.01\linewidth}
\begin{minipage}[b]{.49\linewidth}
\centering
\Large

\blue 

\underline{Relativistic mechanics}


~

$\frac{ \textstyle{dm\vv}}{\textstyle{dt}} = \F $, ~ ~   ~ $m$ varies with $\vv$


\end{minipage}

~

\nin\ These two similar forms of the differential equation that governs charged particle motion state that the motion is 
defined by a second order differential equation. 

~


\nin\  From a mathematical viewpoint, this has the consequence that the motion is considered as defined by 


~

- the knowledge of the forces that intervene

~

-  the knowledge of the initial state of the particle $m$~: initial position and initial velocity~

in particular,   \textsl{initial acceleration or past motion play no role} 

~

\begin{minipage}[b]{.49\linewidth}
\centering
\Large

\blue 

\underline{Classical mechanics }

~

$\F = m \frac{ \textstyle{d^2 \vec M}}{\textstyle{dt^2}} = m \frac{ \textstyle{d\vv}}{\textstyle{dt}} $ 

~

which one can  write

$\F =   \frac{ \textstyle{d m \vv}}{\textstyle{dt}} =  \frac{ \textstyle{d\vec p}}{\textstyle{dt}}  $ 

~

with $\vec p = m \vv$ the impulse, or momentum

~

 $m$  = constant = $m_0$

~
\end{minipage}
\begin{minipage}[b]{.49\linewidth}
\centering
\Large

\blue 

\underline{Relativistic mechanics}


~

$\frac{ \textstyle{dm\vv}}{\textstyle{dt}} = \F $, 

~

 $m$ varies with $\vv$

~

with $\vec p = m \vv$ the impulse, or momentum

~

 $m = m_0 / \sqrt{1- \beta^2}$, with $\beta = v/c$

~

~

\end{minipage}



\clearpage

%\subsection*{\Large The fundamental equation of dynamics}

\begin{figure}

\begin{minipage}[b]{.45\linewidth}
\centering
\Large

\blue 

\underline{Classical mechanics }

~

\underline{Work of the force during the interval  $t_1$ to $t_2$}


\end{minipage}
\begin{minipage}[b]{.54\linewidth}
\centering
\Large

\blue 

\underline{Relativistic mechanics}


~

\underline{Work of the force during the interval  $t_1$ to $t_2$}

\end{minipage}





\begin{center}
\textsl{\Large \bf The variation of the kinetic energy is equal to the work of the forces applied. }
\end{center}


\begin{minipage}[b]{.44\linewidth}
\centering
\Large

\blue 


$\mathcal{T} = \int_{t_1}^{t_2} \F(M,t) . d\vec M ~ \textrm{ with } ~  d\vec M = \vv(t) \, dt $


$= \int_{t_1}^{t_2} m\, \frac{ \textstyle{d\vv}}{\textstyle{dt}} . \vv \, dt $

~

$= \frac{ \textstyle{m}}{\textstyle{2}}\int_{t_1}^{t_2} \frac{ \textstyle{d}}{\textstyle{dt}}(\vv^2) dt $

~

$= \frac{ \textstyle{m}}{\textstyle{2}}\int_{t_1}^{t_2} d(v^2) 
= \frac{ \textstyle{m}}{\textstyle{2}} \times \left[ v^2   \right]_{v_1}^{v_2}$

~

$= W_2 - W_1  $

~

$W =\frac{ \textstyle{1}}{\textstyle{2}} m v^2$ is the kinetic energy.  No need to define the nature of the force (magnetostatic, inductive...)


~

~


The work by $\F$ is $\mathcal{T} = W_2 -W _1 = \frac{ \textstyle{1}}{\textstyle{2}} m (v_2^2 - v_1^2)$ 

~

~


\end{minipage}
\hspace{.01\linewidth}
\begin{minipage}[b]{.04\linewidth}

\blue 

\bf \LARGE $\stackrel{v<< c}{\longleftarrow}$

~

~

\end{minipage}
\begin{minipage}[b]{.52\linewidth}
\centering
\Large

\blue 

~


$\mathcal{T} = \int_{t_1}^{t_2} \F(M,t) . d\vec M ~ \textrm{with} ~  d\vec M = \vv(t) \, dt$

~
 
$= \int_{t_1}^{t_2}  \frac{ \textstyle{d}}{\textstyle{dt}} \left\{ \frac{ \textstyle{m_0\vv}}{\textstyle{\sqrt{1-v^2/c^2}}} \right\} \, \vv \, dt$

~

$= \int_{t_1}^{t_2}  \left( \frac{ \textstyle{m_0\vv . d\vv}}{\textstyle{\sqrt{1-v^2/c^2}}} +\frac{ \textstyle{m_0 \frac{ \textstyle{\vv^2}}{\textstyle{c^2}} \vv . d\vv}}{ \textstyle{ (1-v^2/c^2)^{3/2}} } \right) $


~

$=  \int_{t_1}^{t_2} \frac{ \textstyle{m_0 c^2 \vv . d\vv}}{\textstyle{(1-v^2/c^2)^{3/2}}c^2}  =  \int_{t_1}^{t_2} d \left\{ \frac{ \textstyle{m_0 c^2}}{\textstyle{\sqrt{1-v^2/c^2}}}  \right\} $

~

$= \int_{t_1}^{t_2} d (m c^2)  = (m_2 - m_1 ) c^2$


~

An energy is associated with the mass $m$, 


$E=mc^2$, 


hence a ``rest energy'' $E_0 = m_0 c^2$. 

~

The kinetic energy is defined by $W = E - E_0$

~

The work by $\F$ is $\mathcal{T} = E_2 - E_1 = W_2 -W _1$ 

~

~


\end{minipage}

\end{figure}




\clearpage


\begin{center}

EXERCISE  \label{EX4}

\LARGE 
 \blue 

~

~

~

Show that $\mathcal{T}_{12} = W_2 - W_1 = (m_2 -m_1) c^2 ~ ~ \stackrel{v <\!< c}{\longrightarrow} ~ ~ \frac{ \textstyle{1}}{\textstyle{2}} m_0 (v_2^2 - v_1^2) $

\end{center}









\clearpage

\subsection*{\Large \underline{Deviation of a charged particle in a uniform magnetic field} }


\nib\ The Lorentz force equation~: $  \vec F = q (\vec E + \vec v \times \vec B)$ is reduced to 
{\bf \fbox{$  \vec F = q \, \vec v \times \B$} }

~

\nib\ Remember that the fundamental relation of dynamics yields, 

~

$m_0 \frac{ \textstyle{d\vv}}{\textstyle{dt}} = q \, \vec v \times \B$  in ``classical mechanics'' ($v <\! < c$). 

~

$ \frac{ \textstyle{dm\vv}}{\textstyle{dt}} = q \, \vec v \times \B $ in ``relativistic mechanics'' 
(when $v$ is no longer negligible compared to velocity of light).

~

\nib\  Remember also  that  $\B$ does not work, it cannot induce a change in energy,  the 
velocity and the mass are constant~: 

~

Lorentz relativistic factor $\gamma = 1/\sqrt{1 - v^2/c^2} = $ constant. 

~

The relativistic mass  $m = \gamma m_0$  ~ is  constant. 

~

As a consequence, both classical and relativistic equations  can be written under the form 

~

\begin{center}

\fbox{$ m \frac{\textstyle{d \vv}}{\textstyle{dt}} = q \, \vec v \times \B $ }
\end{center}

\clearpage

\nib\ Only basic considerations will be introduced in the present chapter, 
we will have many occasions to sophisticate things further 
later during the lecture~: 

so, for the moment, we simplify the problem  by taking $\vv_0 $ orthogonal to $\B$. 

~


\begin{minipage}[b]{.65\linewidth}
%\centering
\Large

\nib\ We  simplify the notations, without loss in the generality,  
by taking $\B$ ``vertical''~:   $\B // (y)$. 

 
~

As a consequence the initial velocity is contained in the ``bending plane'', 
which  often happens to be the  ``horizontal plane'', $\vv_0 \in (Osx)$. 

~

~

\nib\ Projection of $ m \frac{\textstyle{d \vv}}{\textstyle{dt}} = q \, \vec v \times \B $ onto the axes yields

~

~

$ m \left| 
\begin{array}{l} 
\frac{ \textstyle{d^2s}}{\textstyle{dt}} \\
\frac{ \textstyle{d^2x}}{\textstyle{dt}}  \\
\frac{ \textstyle{d^2y}}{\textstyle{dt}} \\
\end{array} 
\right. 
= 
q \left| 
\begin{array}{l} 
\dot s    \\
\dot x \\
\dot y \\
\end{array} 
\right. 
\times
\left| 
\begin{array}{l} 
0    \\
0 \\
B_y \\
\end{array}
\right. 
= 
q \left| 
\begin{array}{lr} 
\dot x B_y    \\
-\dot s B_y  \\
0           \\
\end{array} 
\right. 
$    ~ ~ ~ (we introduced $\frac{\textstyle{d()}}{\textstyle{dt}} = \dot {()}$)

\end{minipage}
%\hspace{5mm}
\hfill
\begin{minipage}[b]{.27\linewidth}
\centering
\Large

\blue 
We consider the usual  frame, a direct triedra  $(s,x,y)$. 

We take $\B$  oriented parallel to $(y)$. 

~

\includegraphics[width=6cm]{./beamOptics_Figs/frameB.eps}

\end{minipage}




\clearpage 


\begin{minipage}[b]{.50\linewidth}

\nin Let's now  introduce the ``precession frequency''  
$$ \omega = \frac{\textstyle{qB_y}}{\textstyle{m}}$$
we then get~: ~ ~ ~  

~

\begin{center}
$\left| 
\begin{array}{lr} 
\frac{\textstyle{d^2s}}{\textstyle{dt^2}} = \omega \dot x     & (1)\\
\frac{\textstyle{d^2x}}{\textstyle{dt^2}} = - \omega \dot s    & (2)\\
\frac{\textstyle{d^2y}}{\textstyle{dt^2}} = 0             & (3)\\
\end{array} 
\right.$
\end{center}


\end{minipage}
%\hspace{5mm}
\hfill
\begin{minipage}[b]{.42\linewidth}
\centering
\Large
\blue
\begin{center}
$\left| 
\begin{array}{lr} 
\omega ~ \textrm{is also known as }\\
\textrm{the ``cyclotron frequency''} \\
\textrm{i.e., the angular velocity } \frac{\textstyle{d\theta}}{\textstyle{dt}} \\
\textrm{of a particle in a cyclotron } \\
\textrm{accelerator. } \\
\textrm{Note that} ~ \omega \textrm{ does not}  \\
\textrm{ depend on the radius of the circular }  \\
\textrm{trajectory~: same period }    \\
\textrm{$T=2\pi/\omega$ to perform   }  \\
\textrm{one turn ($\theta = 2\pi$), whatever the radius. }  
\end{array} 
\right.$
\end{center}



\end{minipage}





\clearpage 

{\centering

\Large 


{\blue 

EXERCISE    \label{EX8}

~


A magnet is designed for a proton with velocity 0.2c to perform precession at a rate of $10^{-6}$~second per turn. 

~

What magnetic field value is needed~?  

~

What is  the radius of the proton orbit in that field~?


~

}


}



\clearpage 


\paragraph{\underline{1st integration}}

Equations (1)-(3) cannot be solved independently, they are coupled~: $\dot x$ appears in Eq.~(1) 
whereas $\dot s$ appears in Eq.~(2).

~

{\blue However a first integration is possible and will allow uncoupling the variables~: }


$$
\frac{\textstyle{dv}}{\textstyle{dt}} = 
\left| 
\begin{array}{lr} 
\frac{\textstyle{d^2s}}{\textstyle{dt^2}} = \omega \dot x     & (1)\\
\frac{\textstyle{d^2x}}{\textstyle{dt^2}} = - \omega \dot s    & (2)\\
\frac{\textstyle{d^2y}}{\textstyle{dt^2}} = 0             & (3)\\
\end{array} 
\right.
~ ~ ~ \Rightarrow ~ ~ ~ 
\left| 
\begin{array}{lcrccl} 
 \dot s - \dot s_0 &=&  & \omega (x - x_0)   \\
 \dot x - \dot x_0 &=& -& \omega (s - s_0)  \\
 \dot y - \dot y_0 &=&  & 0 \\
\end{array} 
\right.$$

~

\nin We now introduce the initial conditions~:  ~ ~ ~ $s_0 = 0$, $x_0 = 0$, $\dot y_0 = 0$, \hfill ~

and thus get the first integrals
   \hfill \raisebox{-30mm}[0mm][0mm]{\includegraphics[width=6cm]{./beamOptics_Figs/frameB.eps}}



$$\left| 
\begin{array}{lcrccl} 
\dot s = \dot s_0 + \omega x  & ~ ~ ~ ~ ~ ~   (1') \\
\dot x = \dot x_0 - \omega s  & ~ ~ ~ ~ ~ ~   (2') \\
\dot y =0                      & ~ ~ ~ ~ ~ ~   (3') \\
\end{array} 
\right.$$

~

\nin Re-introducing these first integrals into Eqs.~(1)-(3) then gives 


$$\left| 
\begin{array}{lcclrcccrccr} 
\frac{\textstyle{d^2s}}{\textstyle{dt^2}} &=&  & \omega (\dot x_0 -\omega s)  & ~ ~ ~ ~ ~ & \it i.e., & ~ ~ ~ ~ ~ &  \frac{\textstyle{d^2s}}{\textstyle{dt^2}} + \omega^2 s &=&  & \omega \dot x_0 & ~ ~ ~ ~ ~ ~   (1'')\\
\frac{\textstyle{d^2x}}{\textstyle{dt^2}} &=& -& \omega (\dot s_0 + \omega x) && \it i.e., &  & \frac{\textstyle{d^2x}}{\textstyle{dt^2}}  +  \omega^2 x &=& -& \omega \dot s_0  &  (2'')\\
\frac{\textstyle{d^2y}}{\textstyle{dt^2}} &=&  &  0                           &&           &  &                                                          & &  &   &  (3'') \\
\end{array} 
\right.$$





\clearpage 


\paragraph{\underline{Solving (3'')}} ~ ~ ~ 

~

Integration of differential equation (3'') is straightforward~: 


$$\frac{\textstyle{d^2y}}{\textstyle{dt^2}} = 0 \Rightarrow \frac{\textstyle{dy}}{\textstyle{dt}}  =  \dot y_0  ~ , ~ ~  y =  \dot y_0 \, t \, + \, y_0  $$

Given the initial conditions ~ ~  $\dot y_0 = 0$, ~ ~  $y_0 = 0 $, ~ ~ one gets 

{\blue 
$$ \fbox{y=0} $$ 


\nin \textsl{the motion stays in the $(Osx)$ plane. }
}

~


\clearpage 


\paragraph{\underline{Solving the equations of motion (1''), (2'')}} ~ ~ ~ 

$$\left| 
\begin{array}{lcclrccc} 
  \frac{\textstyle{d^2s}}{\textstyle{dt^2}} + \omega^2 s &=&  & \omega \dot x_0 & ~ ~ ~ ~ ~ ~   (1'')\\
\frac{\textstyle{d^2x}}{\textstyle{dt^2}}  +  \omega^2 x &=& -& \omega \dot s_0  &  (2'')
\end{array} 
\right.$$

~

~

\nin Integration of (1''), (2'') resorts to the regular techniques for solving a second order differential equation of 
the form~: 

$$\frac{\textstyle{d^2z}}{\textstyle{dt^2}} + K z   = C ~, ~ ~ ~  ~ \textrm{with C a constant}, ~  z ~ \textrm{stands for either } ~ s ~ \textrm{or} ~  x$$


~


\nin The general solution is the superimposition of the general solution of the homogeneous equation, right hand side zero~:
$$\frac{\textstyle{d^2z}}{\textstyle{dt^2}} + K z   = 0 ~ ~ ~ ~ ~ ~ ~ ~ (4)$$

~

\nin with a particular solution of 
$$\frac{\textstyle{d^2z}}{\textstyle{dt^2}} + K z   = C ~ ~ ~ ~ ~ ~ ~ ~ (5)$$




\clearpage 



{\blue 

\rule{24cm}{0.5mm}

~

A mathematical parenthesis~: 

~

\nib\ General solution of ~  ~ $\frac{\textstyle{d^2z}}{\textstyle{dt^2}} + K z   = 0$~: 

$$\left| 
\begin{array}{lclr} 
\textrm{if} ~ K  = 0 & : & z = A t + B  & \hspace{3cm}  \textrm{\nib\ A and B are integration constants}       \\
\textrm{if} ~ K  < 0 & : & z = A \cosh\sqrt{-K} t + B \sinh\sqrt{-K} t   &  \textrm{that depend on initial conditions}    \\
\textrm{if} ~ K  > 0 & : & z = A \cos\sqrt{K} t  + B \sin\sqrt{K} t   &\textrm{\nib\ } \left( \begin{array}{c} \cosh \\ \sinh \end{array} (x) = \frac{\textstyle{e^x \pm e^{-x}}}{\textstyle{2}} \right)   
\end{array} 
\right.$$

~

\nib\ Particular solution of ~  ~  $\frac{\textstyle{d^2z}}{\textstyle{dt^2}} + K z   = C$~: 

$$\left| 
\begin{array}{lcl} 
\textrm{if} ~ K     = 0 & : & z = C    \frac{\textstyle{t^2}}{\textstyle{2}}    \\
\\[-1ex]
\textrm{if} ~ K  \neq 0 & : & z =  \frac{\textstyle{C}}{\textstyle{K}}    \\
\end{array} 
\right.$$


~

\nib\ Hence the general solution of ~  ~ $\frac{\textstyle{d^2z}}{\textstyle{dt^2}} + K z   = C$~: 

~

$$\left| 
\begin{array}{lclr} 
\textrm{if} ~ K  = 0 & : & z =  C    \frac{\textstyle{t^2}}{\textstyle{2}}   + A t + B        \\
\textrm{if} ~ K  < 0 & : & z = A \cosh\sqrt{-K} t + B \sinh\sqrt{-K} t +  \frac{\textstyle{C}}{\textstyle{K}}    \\
\textrm{if} ~ K  > 0 & : & z = A \cos\sqrt{K} t  + B \sin\sqrt{K} t +  \frac{\textstyle{C}}{\textstyle{K}}    \\
\end{array} 
\right.$$

~

\rule{24cm}{0.5mm}

}




\clearpage 

{\blue 
EXERCISE     \label{EX9}

~

\nib\ We consider  $\frac{\textstyle{d^2z}}{\textstyle{dt^2}} + K z   = 0$

~

Prove that 
$$\left| 
\begin{array}{lclr} 
\textrm{if} ~ K  = 0 & : & z = A t + B  & ~ ~ ~ ~ \textrm{A and B integration constants}       \\
\end{array} 
\right.$$

~

Prove that 
$$\left| 
\begin{array}{lcl} 
\textrm{if} ~ K  > 0 & : & z = A \cos\sqrt{K} t  + B \sin\sqrt{K} t   
\end{array} 
\right.$$
}





\clearpage

\nin \underline{Back now to our earlier system, }


~

$$\left| 
\begin{array}{lcclrcccrccr} 
  \frac{\textstyle{d^2s}}{\textstyle{dt^2}} + \omega^2 s &=&  & \dot x_0 & ~ ~ ~ ~ ~ ~   (1'')\\
  \frac{\textstyle{d^2x}}{\textstyle{dt^2}}  +  \omega^2 x &=& -& \omega \dot s_0  &  (2'')\\
\end{array} 
\right.$$

~

~

Introducing the initial conditions, at $t=0$~:  $s_0=0$, $x_0=0$, $\dot s = \dot s_0$, $\dot x = \dot x_0$
   \hfill \raisebox{-40mm}[0mm][0mm]{\includegraphics[width=6cm]{./beamOptics_Figs/frameB.eps}}

~

we get 
$$\left| 
\begin{array}{l} 
 s = -\frac{\textstyle{\dot x_0}}{\textstyle{\omega}} \cos\omega t + \frac{\textstyle{\dot s_0}}{\textstyle{\omega}}  \sin\omega t 
+ \frac{\textstyle{\dot x_0}}{\textstyle{\omega}} \\
 x = \frac{\textstyle{\dot s_0}}{\textstyle{\omega}} \cos\omega t  + \frac{\textstyle{\dot x_0}}{\textstyle{\omega}}  \sin\omega t 
- \frac{\textstyle{\dot s_0}}{\textstyle{\omega}}\\
\end{array} 
\right.$$

~

~

We get the trajectory by eliminating the time $t$ between these equations, which yields, 

~

~

$\left| 
\begin{array}{l} 
 \cos\omega t  = 1 + \frac{\textstyle{\omega}}{\textstyle{\dot s_0^2+\dot x_0^2}}   (\dot s_0 x - \dot x_0 s)\\
 \sin\omega t  = \frac{\textstyle{\omega}}{\textstyle{\dot s_0^2+\dot x_0^2}}   (\dot s_0 s + \dot x_0 x) \\
\end{array} 
\right.$   which lends itself to  ~ ~ $\cos^2 + \sin^2 = 1$, thus yielding 


~

~

\hfill \fbox{$\mathbf{ (s - \frac{\textstyle{\dot x_0}}{\textstyle{\omega}})^2 + (x + \frac{\textstyle{\dot s_0}}{\textstyle{\omega}})^2 
 = \frac{\textstyle{\dot s_0^2 + \dot x_0^2}}{\textstyle{\omega^2}}}  $}  \hfill ~




\clearpage




\fbox{$ (s - \frac{\textstyle{\dot x_0}}{\textstyle{\omega}})^2 + (x + \frac{\textstyle{\dot s_0}}{\textstyle{\omega}})^2 
 = \frac{\textstyle{\dot s_0^2 + \dot x_0^2}}{\textstyle{\omega^2}}  $}

~

\nin This is the equation of a circle with radius 
$\rho = \frac{\textstyle{\sqrt{\dot s_0^2 + \dot x_0^2}}}{\textstyle{|\omega|}}= \frac{\textstyle{v_0}}{\textstyle{|\omega|}}$, ~ 
centered at 
$s = \frac{\textstyle{\dot x_0}}{\textstyle{\omega}}, ~ ~  x = -\frac{\textstyle{\dot s_0}}{\textstyle{\omega}}  $. 

~

\hfill DISCUSSION   \hfill ~ ~ 

   \hfill \raisebox{-10mm}[0mm][0mm]{\rotatebox{-70}{\includegraphics[width=6cm]{./beamOptics_Figs/3doigts.eps}}}  \hfill ~


~

\hfill \includegraphics[width=9cm]{./beamOptics_Figs/framebBpos.eps}  \hfill \includegraphics[width=9cm]{./beamOptics_Figs/frameqBneg.eps} \hfill ~

~

{\bf Note}~:  one can write  {\blue \fbox{$B\rho = p/q$}}, ~ given $p=mv$, $v=v_0 = \sqrt{\dot s_0^2 + \dot x_0^2}$ ~ and 
$\omega = qB/m$. 

~

\hfill We call ~ ~  {\blue \fbox{$B\rho$ the rigidity of the particle}}.   \hfill ~ ~ 

~


\end{document}

