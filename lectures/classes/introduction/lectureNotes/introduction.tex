
%Superscripts and subscripts that are words or abbreviations, as in
%\( \pi_{\mathrm{low}} \), should be typed as roman letters; this is
%done as \verb|\( \pi_{\mathrm{low}} \)| instead of \( \pi_{low} \)
%done by \verb|\( \pi_{low} \)|.

%User-defined macros should be placed in the preamble of the chapter,
%and not at any other place in the document. Definitions made using
%the commands \verb|\newcommand,| \verb|\renewcommand,|
%\verb|\newenvironment| or \verb|\renewenvironment| should be used


\chapter[Introduction]{Introduction}\label{chapIntroduction}


\section{Introduction}\label{secIntroduction}

Beyond simple knowledge and concepts, the various 
chapters in this lectures series introduce to the basic principles and formulas attached 
to  accelerator, accelerator physics,  beam dynamics, and other concepts 
which are discussed and manipulated, by means essentially of  computer exercises. 

In a general manner, detailed insight in the theory and technology  of accelerators 
can be found in  series of references found in a bibliography section at the end of each one of these chapters. 
The candidate to these computer exercises  is encouraged to first read these references. 

For each type of accelerator addressed in these lectures, 
the bases will be learned via  practice, namely,  computer simulation exercises. 
The exercises cover 
from the priciples of magnet building bloks (magnets, radio-frequency accelerating systems), 
and their operation (pulsed or fixed-field magnets, frequency-modulated or fixed-frequency RF systems), 
to particle bunch dynamics (closed orbit, focussing, acceleration, 
magnet alignement and field defects, non-linear motion, etc.), 
to particle acceleration outcomes as synchrotron radiation, 
radiation damping, spin, depolarizing resonances, Siberian snakes 
 and other Sokolov-Ternov equilibrium bunch polarization, relativistic life-time and in-flight decay, 
simple space charge models, etc. 



\section{Nomenclature}\label{Nomenclature}


\begin{tabular}{lll}
$c$                    & velocity of light                                          &           \\
$m$                       &     relativistic mass of a  particle                    &            \\
$m_0$                       &     rest mass of a  particle                          &            \\
                  &                     &                 \\
                  &                     &                 \\
                  &                     &                 \\
                  &                     &                 \\
                  &                     &                 \\
                  &                     &                 \\
                  &                     &                 \\
                  &                     &                 \\
                  &                     &                 \\
                  &                     &                 \\
                  &                     &                 \\
                  &                     &                 \\
                  &                     &                 \\
                  &                     &                 \\
                  &                     &                 \\
                  &                     &                 \\
                  &                     &                 \\
                  &                     &                 \\
                  &                     &                 \\
                  &                     &                 \\
                  &                     &                 \\
                  &                     &                 \\
                  &                     &                 \\
                  &                     &                 \\
                  &                     &                 \\
                  &                     &                 \\
                  &                     &                 \\
                  &                     &                 \\
                  &                     &                 \\
                  &                     &                 \\
                  &                     &                 \\
                  &                     &                 \\
                  &                     &                 \\
                  &                     &                 \\
                  &                     &                 \\
\end{tabular}


