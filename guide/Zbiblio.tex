

%\pagestyle{empty}
%\part*{Bibliography}

\pagestyle{myheadings}
\markboth{REFERENCES}{REFERENCES}

\begin{thebibliography}{99}  %%References 
\addcontentsline{toc}{section}{\numberline{}REFERENCES}

\bibitem{Biblio1} 
F.~M\'eot et S.~Val\'ero, 
\textsl{Manuel d'utilisation de  Zgoubi}, 
Rapport IRF/LNS/88-13, CEA Saclay, 1988.  

\bibitem{Biblio2} 
F.~M\'eot and S.~Val\'ero, 
\textsl{Zgoubi users' guide}, 
Int.~Rep. CEA/DSM/LNS/GT/90-05, CEA Saclay (1990) \& TRIUMF report TRI/CD/90-02 (1990). 

\bibitem{Biblio2b} 
F.~M\'eot and S.~Val\'ero, 
\textsl{Zgoubi users' guide - Version 3}, 
Int.~Rep. DSM/LNS/GT/93-12, CEA Saclay (1993). 

\bibitem{Biblio2c} 
F.~M\'eot and S. Valero,  
\textsl{Zgoubi users' guide - Version 4}, 
FNAL Tech.~Rep. FERMILAB-TM-2010 (Aug.~1997),  \& 
Int.~Rep. CEA~DSM~DAPNIA/SEA-97-13, Saclay (Oct.~1997). 

\bibitem{SFZ} http://zgoubi.sourceforge.net/ 

\bibitem{Biblio6} 
F.~M\'eot, 
\textsl{The electrification of Zgoubi}, 
SATURNE report DSM/LNS/GT/93-09, CEA Saclay (1993)~; \\
F.~M\'eot, 
\textsl{Generalization of the Zgoubi method for ray-tracing to include electric fields}, 
NIM A 340 (1994) 594-604. 

\bibitem{Biblio3} 
D.~Carvounas, 
\textsl{Suivi num\'erique de particules charg\'ees dans un sol\'eno\"\i de}, 
rapport de stage, CEA/LNS/GT, 1991. 

\bibitem{Biblio4} 
F.~M\'eot, 
\textsl{Raytracing in 3-D field maps with Zgoubi}, 
report  DSM/LNS/GT/90-01, CEA Saclay, 1990.

\bibitem{Biblio5} 
G.~Leleux, 
\textsl{Compl\'ements sur la physique des acc\'el\'erateurs}, 
cours du DEA Grands-Instruments, Univ.~Paris-VI,  report  IRF/LNS/86-101, CEA Saclay, March 1986. 

\bibitem{Biblio7} 
F.~M\'eot, 
\textsl{A numerical method for combined spin tracking and raytracing of charged particles}, 
NIM \textbf{A313} (1992) 492, and proc. EPAC (1992) p.747. 

\bibitem{polarNuFact}  
D.~J.~Kelliher et al.,
\textsl{Muon decay ring study}, 
Procs. EPAC08 Conf., Genoa, Italy (2008). 

\bibitem{polarSuperB}  
F.~M\'eot, N.~Monseu, 
\textsl{Lattice Design and Study Tools Regarding the Super-B Project, }
Procs. IPAC10 Conf., Kyoto, Japan (2010). 

\bibitem{polarAGS}  
F.~M\'eot, 
\textsl{Spin tracking simulations in AGS based on ray-tracing methods, }
Tech.~Note C-AD/AP/452, BNL C-AD (2009)~; \\ 
F.~M\'eot, 
\textsl{Zgoubi-ing AGS~: spin motion with  snakes and jump-quads,   } 
Tech.~Note C-AD/AP/453, BNL~C-AD~(2009). 

\bibitem{polarRHIC}  
F.~M\'eot, M.~Bai, V.~Ptitsyn, V.~Ranjbar, 
\textsl{Spin Code Benchmarking at RHIC,} 
Procs. PAC11 Conf., New~York (2011). 

\bibitem{polarPEDM}  
F.~M\'eot, 
\textsl{Raytracing Based Spin Tracking, } 
EDM Searches at Storage Rings Workshop, 
ECT - Center for Studies in Nuclear Physics and related Areas, 
Trento, Italy (Oct.~1-5, 2012). 


\bibitem{Biblio8} V. Bargmann, L. Michel, V.L. Telegdi,
\textsl{Precession of the polarization of particles moving in a homogeneous electromagnetic field}, 
Phys. Rev. Lett. 2 (1959) 435.
%%%V.~Bargmann \textsl{et al.}, Phys.\ Rev.\ Lett. \textbf{2} (1959) 435.

\bibitem{FMSEA-00-01} F.~M\'eot and J.~Payet, 
\textsl{Numerical tools for the simulation of synchrotron radiation  
loss and induced dynamical effects in high energy transport lines},  
Report DSM/DAPNIA/SEA-00-01, CEA Saclay (2000). 

\bibitem{SRLossGL} 
G.~Leleux et al., 
\textsl{SR perturbations in long transport lines, } 
IEEE 1991 Part Acc. Conf., San Francisco (May 1991). 

\bibitem{DYNAC} 
P.~Lapostolle, F.~M\'eot, S.~Valero, 
\textsl{A new dynamics code  DYNAC for electrons, protons and heavy ions in LINACS with long accelerating elements, } 
1990 LINAC Conf., Albuquerque, NM, USA. 

\bibitem{ELFE} 
\textsl{Electron Lab for Europe, } 
Blue Book, CNRS-IN2P3 (1994). 

\bibitem{SRLossBench} 
F.~M\'eot, 
\textsl{Benchmarking stepwise ray-tracing in rings in presence of radiation damping, }
Procs. PAC11 Conf., New~York (2011). 

\bibitem{VOKostroun} 
V.~O.~Kostroun, 
\textsl{Simple numerical evaluation of modified Bessel functions and integrals [...]}, 
NIM 172 (1980) 371-374. 

\bibitem{FMSL/94-22} 
F.~M\'eot, 
\textsl{Synchrotron radiation interferences at the LEP miniwiggler}, 
Report CERN SL/94-22 (AP), 1994. 

\bibitem{FMLPYellow} 
L. Ponce,  R. Jung, F. M\'eot, 
\textsl{LHC proton beam diagnostics using synchrotron radiation},  
Yellow Report CERN-2004-007. 

\bibitem{PALowFreq} 
F.~M\'eot, 
\textsl{A theory of low frequency far-field synchrotron radiation}, 
 Particle Accelerators Vol~62, pp.~215-239  (1999). 

\bibitem{Albert} 
Albert~Hofmann, 
\textsl{The Physics of Synchrotron Radiation}, 
Cambridge University Press, May 13, 2004. 

\bibitem{Biblio9} B.~Mayer, personal communication,
CEA Saclay, Laboratoire National SATURNE, 1990. 

\bibitem{Biblio10} 
L.~Farvacque \textsl{et al.}, 
\textsl{Beta user's guide}, 
Note ESRF-COMP-87-01, 1987~; \\
J.~Payet, IRF/LNS, CEA Saclay, private communication~; \\
J.M.~Lagniel, 
\textsl{Recherche d'un optimum}, 
Note IRF/LNS/SM 87/48, CEA Saclay 1987.

\bibitem{NelderMead} 
Installed by J.~S.~Berg, BNL (2007). 
\emph{Cf.} \textsl{Detection and remediation of stagnation in the 
Nelder-Mead algorithm using a sufficient decrease condition}, 
  C.~T.~Kelley, Siam J. Optim., Vol. 10, No. 1, pp. 43-55. 

\bibitem{Coupling} 
F.~Desforges, 
\textsl{Implementation of a coupled treatment of the one-turn mapping in the ray-tracing code \zgoubi, }
C-AD~Note C-A/AP/461, BNL, Sept~2012.

\bibitem{Biblio11} 
F.~M\'eot and N.~Willis, 
\textsl{Raytrace computation with Monte Carlo simulation of particle decay}, 
internal report CEA/LNS/88-18 CEA Saclay, 1988. 

\bibitem{EJBleser} 
E.J.~Bleser, 
\textsl{The parameters of the bare AGS,}
 Tech.~Note AGS/AD?Tech.~Note~430, March~15, 1996.

\bibitem{MADXAGSModel} These transfer functions have been copied from the MADX model of the AGS.

\bibitem{Biblio12} 
H.A.~Enge, 
\textsl{Deflecting magnets}, in \textbf{Focusing of Charged Particles}, 
ed.~A.~Septier, \textbf{Vol.}  \textbf{II}, pp 203-264, Academic
Press Inc., 1967.

\bibitem{Biblio13} 
P.~Birien et S.~Val\'ero, 
\textsl{Projet de spectrom\`etre magn\'etique \`a haute r\'esolution pour ions lourds}, 
\textbf{Section IV} p.62, Note CEA-N-2215, CEA Saclay, mai 1981.



\newpage



\bibitem{BBSW} Files developed by Simon~White, January 2012. Including beam-beam spin kick after Ref.~\cite{YKBatyginSpin}. 

\bibitem{YKBatyginSpin} 
Y.~K.~Batygin, 
\textsl{Spin depolarization due to beam-beam collisions, } 
Phys.~Rev.~E, Vol.~58,  1, July~1998. 

\bibitem{reportNIMFFAG} 
F.~Lemuet, F.~M\'eot, 
\textsl{Developements in the ray-tracing code Zgoubi for 6-D multiturn tracking in FFAG rings}, 
NIM~A \textbf{547} (2005) 638-651. 

\bibitem{reportICFAFFAG} F.~M\'eot,  
\textsl{6-D beam dynamics simulations in FFAGs using the ray-tracing code Zgoubi}, 
ICFA Beam Dyn.Newslett.43:44-50 (2007).

\bibitem{theseLemuet} 
Franck~Lemuet, 
\textsl{Collection and muon acceleration in the neutrino factory project}, 
PhD thesis, Paris~KI University, April 2007. 

\bibitem{Biblio14} 
V.~M.~Kel'man and S.~Ya.~Yavor, 
\textsl{Achromatic quadrupole electron lenses, } 
Soviet Physics - Technical Physics, vol.~6, No~12, June 1962~; \\
S.~Ya.~Yavor \textsl{et als.}, 
\textsl{Achromatic quadrupole lenses}, 
NIM \textbf{26} (1964) 13-17. 

\bibitem{Biblio16} 
A.~Septier, 
\textsl{Cours du DEA de physique des particules, optique corpusculaire,}  
Universit\'e d'Orsay, 1966-67, pp.~38-39.  

\bibitem{Karets} 
S.~P.~Karetskaya et als., 
\textsl{Mirror-bank energy analyzers, in Advances in electronics and electron physics}, 
Vol. 89, Acad. Press (1994) 391-491. 

\bibitem{Biblio15} 
A.~Septier et J.~van Acker, 
\textsl{Les lentilles quadrupolaires \'electriques}, 
NIM \textbf{13} (1961) 335-355~; \\
Y.~Fujita and H.~Matsuda, 
\textsl{Third order transfer matrices for an electrostatic quadrupole lens,} 
NIM \textbf{123} (1975) 495-504. 

\bibitem{reportNIMFFAGSPI} 
J.~Fourrier, F.~Martinache, F.~M\'eot, J.~Pasternak, 
\textsl{Spiral FFAG lattice design tools,  application to 6-D tracking in a proton-therapy class lattice}, 
NIM~A~589 (2008). 

\bibitem{repDapniaEMMA} 
F.~M\'eot, 
\textsl{Tracking studies regarding EMMA FFAG project, }
Internal report CEA DAPNIA-06-04 (2006). 

\bibitem{EMMAIPAC10} 
J.~S.~Berg et al., 
\textsl{Recent developments on the EMMA on-line commissioning software,} 
Procs. IPAC10 Conf., Kyoto, Japan (2010). 

\bibitem{EMMA} 
S. Machida et al., 
\textsl{Acceleration in the linear non-scaling fixed-field alternating-gradient accelerator EMMA, }
Nature Physics, vol.~8, March 2012.

\bibitem{Pavel} Installed by Pavel~Akishin, JINR, Dubna, 1992.

\bibitem{Biblio17} 
M.W.~Garrett, 
\textsl{Calculation of fields  [...] by elliptic integrals}, 
J.~Appl.~Phys., \textbf{34}, 9, sept.~1963.  

\bibitem{Biblio18} 
P.F.~Byrd and M.D.~Friedman, 
\textsl{Handbook of elliptic integrals for engineers and scientists,} 
pp.~282-283, Springer-Verlag, Berlin, 1954.  

\bibitem{SPINR} Installation by M.~Bai, BNL, 2009.  

\bibitem{Biblio19} 
A.~Tkatchenko, 
\textsl{Computer program UNIPOT}, 
SATURNE, CEA Saclay, 1982. 

\bibitem{RAYTRACE} 
S.B. Kowalski, H.A. Enge, 
The ion-optical program raytrace, 
NIM~A258 Vol.~3 (1987) 407. 

\bibitem{RAYTRACECoeffs} 
N. Tsoupas, private communication, Brookhaven National Laboratory, Oct 2015. 

\bibitem{BiblioPlot} 
J.L.~Chuma, 
\textsl{PLOTDATA}, 
TRIUMF Design Note TRI-CO-87-03a.

\bibitem{AGSModel} 
F.~M\'eot et la., 
A model of the AGS in the ray-tracing code Zgoubi, 
Tech. Note C-A/AP/***, 2013. 

\bibitem{Spes2} 
J.~Thirion et P.~Birien, 
\textsl{Le spectrom\`etre II, Internal Report DPh-N/ME}, 
CEA Saclay, 23 D\'ec. 1975~; \\
H.~Catz, 
\textsl{Le spectrom\`etre SPES~II}, Internal Report DPh-N/ME, CEA Saclay, 1980~; \\
 A.~Moalem, F.~M\'eot, G.~Leleux, J.P.~Penicaud, A.~Tkatchenko, P.~Birien,  
\textsl{A modified QDD spectrometer for $\eta$ meson decay measurements,  } 
NIM A289 (1990) 168-175 

\bibitem{BNL}  
P.~Pile, I-H.~Chiang, K.~K.~Li, C.~J.~Kost, J.~Doornbos, F.~M\'eot et als., 
\textsl{A two-stage separated 800-MeV/c Kaon beamline}, 
 TRIUMF and BNL Preprint (1997).

\bibitem{ZgCern} 
F.~M\'eot, 
\textsl{The raytracing code Zgoubi}, CERN SL/94-82 (AP) (1994), 
3rd Intern. Workshop on Optimization and Inverse Problems 
in Electromagnetism, CERN, Geneva, Switzerland, 19-21 Sept. 1994. 

\bibitem{Grorud} 
E.~Grorud, J.L.~Laclare, G.~Leleux, 
\textsl{R\'esonances de d\'epolarisation dans SATURNE~2}, 
Int. report GOC-GERMA 75-48/TP-28, CEA Saclay (1975), 
and, Home Computer Codes POLAR and POPOL, IRF/LNS/GT, CEA Saclay (1975).

\bibitem{Froissart} 
M.~Froissart et R.~Stora, 
\textsl{D\'epolarisation d'un faisceau de protons polaris\'es dans un synchrotron},
 NIM 7 (1960) 297-305.

\bibitem{HAMEL} 
J.L.~Hamel, 
\textsl{mini graphic library LIBMINIGRAF}, 
CEA-DSM, Saclay, 1996.

\bibitem{LHCb10} 
F.~M\'eot∗ e , A.~Par\'{\i}s, 
\textsl{Concerning effects of fringe fields and longitudinal distribution of b10 in
low-$\beta$ regions on dynamics in LHC, } 
report FERMILAB-TM-2017, August 23, 1997. 

\bibitem{EIC14} 
F. M\'eot, 
\textsl{End-to-end 9-D polarized bunch transport in FFAG eRHIC}, 
EIC14 workshop, http://www.jlab.org/conferences/eic2014/ (2014). 


\end{thebibliography}



