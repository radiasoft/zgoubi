\documentclass[10pt]{article}

%\usepackage{draftcopy}
%\usepackage[draft]{graphicx}
\usepackage{graphicx}
%\usepackage{wrapfig}
\usepackage{amssymb}
\usepackage{lscape}
\usepackage{times}
\usepackage{color}  % For \textcolor and \color % Ex. : \textcolor{red}{Text colored with} ; {\color{red}Text colored with}
\usepackage{soul}   % For \hl{ highlighted text} ; \sethlcolor{colorname}
%\usepackage[table]{xcolor}
\usepackage{xcolor,colortbl}

%----------------------
%   portrait
%\oddsidemargin =-0.4in
%\evensidemargin=-0.4in
%\textwidth=6.8in
%\textheight=8in
%\topmargin=-1.in
%\footskip=0in
%   landscape
 \oddsidemargin =-0.8in                            
 \evensidemargin=-0.8in                            
 \textwidth=8.2in              
 \textheight=10.1in  
 \topmargin=-.9in  % laptop
% \topmargin=-.15in  % owl
 \footskip=0in       
%----------------------

\newcommand{\Br}{\ensuremath{B\!\rho}}
\newcommand{\bull}{\ensuremath{\bullet~}}
\newcommand{\Bz}{\ensuremath{{B_z}}}
\newcommand{\CE}{concentration ellipse}
\newcommand{\CES}{CE-$\mathcal{S}$}
\newcommand{\com}{{center of mass}}
\newcommand{\dEE}{\small \ensuremath{\frac{dE}{E}}}
\newcommand{\dip}{\textit{ DIPOLE}}
\newcommand{\eg}{\textsl{e.g.}}
\newcommand{\EFB}{\ensuremath{E\!F\!B}}
\newcommand{\EFBs}{\ensuremath{E\!F\!B\!s}}
\newcommand{\ffag}{\textit{ FFAG}}
\newcommand{\hel}{\ensuremath{\mathbf{^3 H\! e ^{2+}}}}
\newcommand{\hbrk}{\hfill \break}
\newcommand{\x}{\ensuremath{x}} 
\newcommand{\xp}{\ensuremath{{x'}}}
\newcommand{\y}{\ensuremath{y} }
\newcommand{\yp}{\ensuremath{{y'}}}
\newcommand{\dl}{\ensuremath{{\delta l}}} 
\newcommand{\lab}{\ensuremath{lab. frame}}
\newcommand{\MC}{Monte~Carlo}
\newcommand{\nib}{\noindent \ensuremath{\bullet~}}
\newcommand{\nin}{\noindent~}
\newcommand{\p}{\ensuremath{\mathbf{p}}}
\newcommand{\pp}{$p\! \! \uparrow$}
\newcommand{\rms}{\ensuremath{rms}}
\newcommand{\SRl}{SR loss}
\newcommand{\Sx}{\ensuremath{\mathcal{S}_x}}
\newcommand{\Sy}{\ensuremath{\mathcal{S}_y}}
\newcommand{\Sz}{\ensuremath{\mathcal{S}_z}}
\newcommand{\wrt}{{with respect to}}
\newcommand{\Y}{\ensuremath{Y} }
\newcommand{\Yp}{\ensuremath{{Y'}}}
\newcommand{\z}{Zgoubi}


\newcommand{\C}{\ensuremath{\mathcal{C}}}
\newcommand{\D}{\ensuremath{\mathcal{D}}}
\newcommand{\HH}{\ensuremath{\mathcal{H}}}
\newcommand{\bHH}{\bar \HH}
\newcommand{\LL}{\ensuremath{\mathcal{L}}}
\newcommand{\zg}{Zgoubi}

\newcommand{\black}{\color{black}}
\newcommand{\red}{\color{red}}
\newcommand{\green}{\color{green}}
\newcommand{\blue}{\color{blue}}
\newcommand{\BurntOrange}{\color{BurntOrange}}

\newcommand{\hlyell}[1]{{\sethlcolor{yellow}\hl{#1}}}
\newcommand{\hlcyan}[1]{{\sethlcolor{cyan}\hl{#1}}}

\newcolumntype{a}{>{\columncolor{gray!40!white}}c}
\newcolumntype{b}{>{\columncolor{blue!10!white}}c}

\definecolor{yell80}{rgb}{1,1,.8}
\definecolor{yell60}{rgb}{1,1,.6}
\definecolor{yell40}{rgb}{1,1,.4}
\definecolor{yell20}{rgb}{1,1,.2}
\definecolor{bluelight}{rgb}{.75,.946,1.}

\newcommand{\referenceA}{\rm }
\newcommand{\referenceB}{\rm F. M\'eot, Dec. 2016}
\newcommand{\referenceC}{\rm }

\markboth{\small  \referenceA ~ ~   \referenceB ~ ~  \referenceC \hfill }
         {\small  \referenceA ~ ~   \referenceB ~ ~  \referenceC \hfill }



\begin{document}
\landscape

\Large  \bf

\thispagestyle{myheadings}


\title{
%\bf  \includegraphics{nufact4c_small.eps}
  \hfill  \referenceA \\
 \hfill \referenceB  \\
 \hfill \referenceC  \\
~ ~ ~ \\
{\bf \blue
\Huge Effect of snakes on tunes and chromas, at injection
} 
}

\date{}

\pagestyle{headings}

\maketitle

\tableofcontents

 


\clearpage

\nib RHIC 4-module snakes are simulated using Ramesh's two 3D OPERA maps of respectively  
right-handed and  left-handed modules. 

~

\nib Tunes and chromas,  snakes off or on, $G\gamma = 45.5$. 

Note~: inkection optics should be used, collision is, instead, for now (optical functions 
plots in slied \# \pageref{slide:optics}). Snakes change the  focusing in IR regions, 
this may have substantial effect on tune. This has to be updated with proper optics.  

~

~

\begin{minipage}{1.\linewidth}

\begin{minipage}{.49\linewidth}
SNAKES OFF
\tiny
\begin{verbatim}
  Reference, before change of frame (part #     1)  : 
   0.  -3.46581390E-06  -1.18678295E-05   0.   0.   3.83384557E+05   1.27982715E+01

                TWISS  parameters,  periodicity  of   1  is  assumed 
                                   -  COUPLED  -

       Beam  matrix  (beta/-alpha/-alpha/gamma) and  periodic  dispersion  (MKSA units)
           1.949062     0.058889     0.000000     0.000000     0.000000     0.001580
           0.058889     0.514847     0.000000     0.000000     0.000000     0.002030
           0.000000     0.000000     2.059936    -0.027239     0.000000     0.000000
           0.000000     0.000000    -0.027239     0.485812     0.000000     0.000000
           0.000000     0.000000     0.000000     0.000000     0.000000     0.000000
           0.000000     0.000000     0.000000     0.000000     0.000000     0.000000

                              Betatron  tunes  (Q1 Q2 modes)

                    NU_Y =  0.68499553         NU_Z =  0.67299602    

                                        ----------------------------------------
                                        -  EDWARDS AND TENG`S PARAMETRIZATION  -
                                        ----------------------------------------

                                                                              MODE 1               MODE 2

      FRACTIONAL PART OF THE BETATRON TUNES IN THE DECOUPLED FRAME:        0.68499553           0.67299602

      EDWARDS-TENG`S PARAMETERS:
                                - ALPHA:                                   -0.05888897           0.02723900
                                - BETA:                                     1.94906212           2.05993600
                                - GAMMA:                                    0.51484655           0.48581216

      HAMILTONIAN PERTURBATION PARAMETERS:

                           - DISTANCE FROM THE NEAREST DIFFERENCE LINEAR RESONANCE:     0.00000000
                           - COUPLING STRENGTH OF THE DIFFERENCE LINEAR RESONANCE:      0.00000000

                           - DISTANCE FROM THE NEAREST SUM LINEAR RESONANCE:            0.00000000
                           - UNPERTURBED HORIZONTAL TUNE:                               0.68499553
                           - UNPERTURBED VERTICAL TUNE:                                 0.67299602

                                   Momentum compaction : 

                              dL/L / dp/p =  1.78866203E-03
     (dp =  1.000000E-04      L(0)   =    3.83385E+05 cm,  L(0)-L(-dp) =    6.85822E-02 cm,  L(0)-L(+dp) =   -6.85669E-02 cm) 

                                   Transition gamma  =  2.36448115E+01

                                   Chromaticities : 

                              dNu_y / dp/p =  2.00068579E+00
                              dNu_z / dp/p =  1.99970750E+00
\end{verbatim}
\end{minipage}
\begin{minipage}{.001\linewidth}
\rule{0.1mm}{11cm}
\end{minipage}
\begin{minipage}{.49\linewidth}
SNAKES ON

\tiny
\begin{verbatim}
  Reference, before change of frame (part #     1)  : 
   0.  -4.56024376E-06   6.31737633E-06  -1.55944093E-06   3.37854334E-05   3.83384968E+05   1.27982854E+01

                TWISS  parameters,  periodicity  of   1  is  assumed 
                                   -  COUPLED  -

       Beam  matrix  (beta/-alpha/-alpha/gamma) and  periodic  dispersion  (MKSA units)
           1.936703     0.050666     0.000000     0.000000     0.000000     0.000149
           0.050666     0.517667     0.000000     0.000000     0.000000     0.001667
           0.000000     0.000000     2.007300    -0.332127     0.000000    -0.000119
           0.000000     0.000000    -0.332127     0.553135     0.000000    -0.001162
           0.000000     0.000000     0.000000     0.000000     0.000000     0.000000
           0.000000     0.000000     0.000000     0.000000     0.000000     0.000000

                              Betatron  tunes  (Q1 Q2 modes)

                    NU_Y =  0.68194766         NU_Z =  0.73614756    


                                        ----------------------------------------
                                        -  EDWARDS AND TENG`S PARAMETRIZATION  -
                                        ----------------------------------------

                                                                              MODE 1               MODE 2

      FRACTIONAL PART OF THE BETATRON TUNES IN THE DECOUPLED FRAME:        0.68194766           0.73614756

      EDWARDS-TENG`S PARAMETERS:
                                - ALPHA:                                   -0.05066621           0.33212699
                                - BETA:                                     1.93670259           2.00729961
                                - GAMMA:                                    0.51766702           0.55313533

      HAMILTONIAN PERTURBATION PARAMETERS:

                           - DISTANCE FROM THE NEAREST DIFFERENCE LINEAR RESONANCE:    -0.05405238
                           - COUPLING STRENGTH OF THE DIFFERENCE LINEAR RESONANCE:      0.00399628

                           - DISTANCE FROM THE NEAREST SUM LINEAR RESONANCE:            0.41809522
                           - UNPERTURBED HORIZONTAL TUNE:                               0.68202142
                           - UNPERTURBED VERTICAL TUNE:                                 0.73607380

                                   Momentum compaction : 

                              dL/L / dp/p =  1.78676165E-03
     (dp =  1.000000E-04      L(0)   =    3.83385E+05 cm,  L(0)-L(-dp) =    6.85099E-02 cm,  L(0)-L(+dp) =   -6.84936E-02 cm) 

                                   Transition gamma  =  2.36573823E+01

                                   Chromaticities : 

                              dNu_y / dp/p =  3.82686455E+00
                              dNu_z / dp/p =  3.96685338E+00
\end{verbatim}

\end{minipage}

~

~

\nib From left to right, tunes change from 

{\blue 
\fbox{uncoupled, NU\_Y =  0.68499553    ~ ~     NU\_Z =  0.67299602    ~ ~  ~ ~  to, weakly coupled, ~ ~ Q1 =  0.68194766  ~ ~ Q2 =  0.73614756   }
}

and chromas increase by ~1.9 unit, from 2 to ~3.9.



\end{minipage}



\clearpage

\nib Just to show that residual orbits in these simulations are negligible~: 

~

~

\begin{minipage}{1.\linewidth}

\mbox{
\begin{minipage}{.49\linewidth}
\nib SNAKES OFF

~

HORIZONTAL :

~

\includegraphics*[bbllx=20,bblly=105,bburx=550,bbury=460,width=.7\linewidth]{residualOrbit_snakesOff.eps} 


~

~

~

~

Initial vertical conditions zero remain zero.

\end{minipage}
\begin{minipage}{.001\linewidth}
\rule{0.1mm}{18cm}
\end{minipage} ~  ~ 
\begin{minipage}{.49\linewidth}
\nib SNAKES ON

~

HORIZONTAL :

~

\includegraphics*[bbllx=20,bblly=105,bburx=550,bbury=460,width=.7\linewidth]{residualOrbit_snakesOn_H.eps} 

~

~

VERTICAL :

~

\includegraphics*[bbllx=20,bblly=105,bburx=550,bbury=460,width=.7\linewidth]{residualOrbit_snakesOn_V.eps} 

\end{minipage}
}

\end{minipage}



\clearpage

\nib Shows Y and Z orbit excursions in the two snakes~: 


~

~

\begin{minipage}{1.\linewidth}
\begin{minipage}{.49\linewidth}
\centering

SNAKE 1 : 

~

\includegraphics*[bbllx=20,bblly=105,bburx=550,bbury=460,width=.9\linewidth]{orbit_snake1_45.5.eps} 

~

\end{minipage}
\begin{minipage}{.49\linewidth}
\centering
SNAKE 2 : 

~

\includegraphics*[bbllx=20,bblly=105,bburx=550,bbury=460,width=.9\linewidth]{orbit_snake2_45.5.eps} 


\end{minipage}

\end{minipage}




\clearpage

\nib Optical functions ($G.gamma=45.5$ -~with collision optics~(!))~:

~

~

\begin{minipage}{1.\linewidth}

\begin{minipage}{.49\linewidth}

\label{slide:optics}

\nib SNAKES OFF

~

\includegraphics*[width=.99\linewidth]{gnuplot_TWISS_btxy_snakesOff.eps} 

~

\includegraphics*[width=.99\linewidth]{gnuplot_TWISS_xy_snakesOff.eps} 


\end{minipage}
\begin{minipage}{.001\linewidth}
\rule{0.1mm}{18cm}
\end{minipage}
\begin{minipage}{.49\linewidth}
\nib SNAKES ON

~

\includegraphics*[width=.99\linewidth]{gnuplot_TWISS_btxy_snakesOn.eps} 

~

\includegraphics*[width=.99\linewidth]{gnuplot_TWISS_xy_snakesOn.eps} 


\end{minipage}

\end{minipage}





\end{document}




